\pdfoutput=1
% In particular, the hyperref package requires pdfLaTeX in order to break URLs across lines.

\documentclass[11pt]{article}

% Change "review" to "final" to generate the final (sometimes called camera-ready) version.
% Change to "preprint" to generate a non-anonymous version with page numbers.
\usepackage[preprint]{acl}

% Standard package includes
\usepackage{times}
\usepackage{latexsym}

% For proper rendering and hyphenation of words containing Latin characters (including in bib files)
\usepackage[T1]{fontenc}
% For Vietnamese characters
% \usepackage[T5]{fontenc}
% See https://www.latex-project.org/help/documentation/encguide.pdf for other character sets

% This assumes your files are encoded as UTF8
\usepackage[utf8]{inputenc}

% This is not strictly necessary, and may be commented out,
% but it will improve the layout of the manuscript,
% and will typically save some space.
\usepackage{microtype}

% This is also not strictly necessary, and may be commented out.
% However, it will improve the aesthetics of text in
% the typewriter font.
\usepackage{inconsolata}

\usepackage{amsmath}
\usepackage{amsfonts}
\usepackage{geometry}
\usepackage[demo]{graphicx}
\usepackage{float}
\usepackage{cleveref}
\usepackage{tabularx}             % For flexible-width columns
\usepackage{booktabs}             % For nicer table rules
\usepackage{array}                % Extended col types if needed

% If the title and author information does not fit in the area allocated, uncomment the following
%
%\setlength\titlebox{<dim>}
%
% and set <dim> to something 5cm or larger.

% Limit the number of unprocessed floats
\setcounter{totalnumber}{10}
\setcounter{topnumber}{5}
\setcounter{bottomnumber}{5}

\title{\textbf{LINKS}: Generate linguistically grounded mnemonic devices for English vocabulary learning with reasoning, multilingual LLMs
}

\newcommand{\links}{\textbf{\textsc{Links}}}
\newcommand{\recorder}{\textregistered}
\newcommand{\studentmodel}{\studentmodel}
\newcommand{\teachermodel}{\teachermodel}
\newcommand{\mnemonic}{$m$}
\newcommand{\mnemonics}{$m$'s}
\newcommand{\vocabulary}{$v$}
\newcommand{\eg}{$e$}

\author{%
  My (Chiffon) Nguyen \\
  College of Computational Sciences \\
  Minerva University \\
  San Francisco, CA 94108 \\
  chiffonng@uni.minerva.edu
}

\begin{document}

\begin{titlepage}
\centering
{\scshape\LARGE Minerva University \par}
\vspace{1cm}
\begin{center}
    \includegraphics[width = 0.4\linewidth]{figures/minerva_logo.pdf}
\end{center}
{\scshape\Large Capstone: Class of 2025 \par}
\vspace{1.5cm}
{\huge\bfseries LINKS: Generate linguistically grounded mnemonic devices for English vocabulary learning with reasoning, multilingual LLMs \par}
\vspace{2cm}
{\scshape\large Tra My Nguyen \par}
submitted in partial fulfillment of the requirements for the degree of Bachelor of Science in Computational Sciences \par
\vspace{2cm}
{\large\itshape Capstone Committee \par}
Dr. Patrick Watson \\
Dr. Philip Sterne

\vfill
{\large \today\par}
\end{titlepage}

\onecolumn
\section*{Executive Summary}
Note: Due to limited compute, some experiments conducted are small-scale and need more data for robust validation and conclusion. However, the codebase is reproducible and scalable when there is more compute.

Tags: computational linguistics, natural language processing, large language model, language education, english as a foreign language, vocabulary acquisition, synthetic data generation.

\clearpage

\tableofcontents
\clearpage

\twocolumn

\maketitle
\begin{abstract}
To acquire advanced vocabulary, English learners often use mnemonic devices, memorable associations linking a new concept to learned concepts to improve memory and recall. Reviewing the literature on mnemonic techniques, we characterize good mnemonics as \textbf{linguistically grounded}, which better link to the target vocabulary, improving long-term retention and linguistic knowledge, especially at advanced levels (CEFR B2+). We investigate whether Large Language Models can consistently help write such effective mnemonics, with three different settings: in-context learning, and reasoning distillation. Concretely, we first measured different prompting strategies with a frontier reasoning model, \teachermodel, and generated \links, a synthetic dataset of 2000 triplets of \textit{reasoning trace, mnemonic, and example sentence} for 2000 vocabulary useful for TOEFL iBT \footnote{Internet-based Test of English as a Foreign Language}, IELTS Academic \footnote{International English Language Testing System}, and SAT\footnote{Scholastic Aptitude Test}. Second, using a subset of \links, we distilled linguistic reasoning from the \textit{teacher model} to the \textit{student model}, \studentmodel\footnote{\url{https://huggingface.co/collections/google/gemma-3-release-67c6c6f89c4f76621268bb6d}} , with online reinforcement learning. The trained, quantized model can be served with a local application such as OpenWebUI (interface) and Ollama (command-line).

Preliminary evaluation shows

The project examplifies that carefully designed NLP systems can generate resources for language learning, either in classroom settings or in self-study. Code, models, and dataset are available\footnote{Links: \url{https://github.com/chiffonng/mnemonic-gen}}.
\end{abstract}

\section{Introduction}
Vocabulary acquisition challenges many English language learners, particularly at upper intermediate to advanced levels where abstract and academic vocabulary predominates. Mnemonics, cognitive tools that help learners create associations between new vocabulary and familiar concepts, serve as valuable tools for enhancing retention and recall. The deeper learners elaborate the connection between the mnemonic and the target vocabulary, the longer and better they can recall the term. However, creating such effective mnemonics demands both linguistic expertise and creative effort, presenting a significant barrier for most learners. Large Language Models (LLMs) have demonstrated capabilities as knowledge bases and creative text generators, suggesting their potential for automated mnemonic generation.

Previous work explored automated mnemonic generation through computational methods using the \textbf{keyword method}, which involves 1) generateing simpler keywords that together sound or look like the target vocabulary and 2) creating memorable explanations that include the vocabulary, the keywords, amd its meaning \citep{atkinsonApplicationMnemonicKeyword1975}. \citet{savvaTransPhonerAutomatedMnemonic2014} and \citet{OzbalAUTOMATION2014} generated keywords of phonetic and orthographic similarities in the native language for foreign language vocabulary, across multiple languages. \citet{LeeSMARTPHONE2023} extended this work and utilized LLMs to produce phonetically similar keywords and visual cues and \citet{LeeEXPLORING2024} prompted LLMs to generate multiple mnemonic candidates and evaluate them based on imageability and coherence. Most recently, \citet{balepurSMART2024} fine-tuned LLaMA-2-70B on 800 crowd-sourced mnemonics and aligned outputs with learner preferences and learning outcomes.

%%%% Figure Highlight the difference between keyword "mnemonic" and "linguistically grounded mnemonics" with an example
%%% (e.g., \textbf{preposterous} can be broken down as pre- (before) + poster (after) + ous. Anything that comes both before and after is preposterous)

Although the keyword method is commonly used and empirically validated in classroom and laboratory contexts \citetext{\citealp{atkinsonApplicationMnemonicKeyword1975}, \citealp{pressleyMnemonicKeywordMethod1982}}, it may lead to longer retrieval time \citep{vanhellKeywordMnemonicsRote1997} and be inadequate for fairly abstract vocabulary \citetext{\citealp{camposLimitationsMnemonicKeywordMethod2003}, \citealp{camposImportanceKeywordGenerationMethod2004a}}. Such methods typically neglect other rich linguistic knowledge embedded in LLMs that could provide diverse mnemonic strategies beyond simple keyword associations. Second, previous works passively deliver generated mnemonics to learners. While \textsc{Smart} \citep{balepurSMART2024} was further trained on learners' preferences, these preferences were aggregated, potentially missing alignment with individual learning styles. Language learners who use self-created mnemonics retain vocabulary more effectively and for longer duration \citep{madanExploringWordMemorability2021}.

Our contributions can be summarized as follows. (\textbf{1}) We demonstrate that LLMs can generate \textbf{linguistically grounded mnemonics}, which emphasizes the importance of linguistic features in creating effective mnemonics, through reasoning and creative writing. (\textbf{2}) We present \links, a synthetic dataset of 2000 triplets of \textit{reasoning trace, mnemonic, and example sentence} for 2000 vocabulary. They can be integrated in a spaced repetition system (SRS) or language learning applications for vocabulary acquisition with better retrieval. (\textbf{3}) We distill the reasoning capabilities of a frontier, reasoning LLM, into a smaller model, \studentmodel, using \citep{DeepSeek-AIDEEPSEEKR12025}. The trained model \links, can be served locally, enabling users to generate mnemonics without relying on external APIs or internet connectivity.

\section{Background}

We assume a background on LLMs, including their transformer-based architecture (\Cref{app:llm-transformer}), in-context learning (\Cref{sec:icl-performance}), and reinforcement learning (full preliminaries are provided in \Cref{app:technicality}). We briefly review the literature on mnemonic devices for vocabulary learning and the use of LLMs in linguistic tasks.

\subsection{Mnemonic devices for vocabulary learning}

\begin{table*}[htb]
\centering
\caption{Examples of feature categories for English words.}
\label{tab:linguistic-features}
\begin{tabularx}{\textwidth}{l >{\raggedright\arraybackslash}X >{\raggedright\arraybackslash}X}
\toprule
\textbf{feature} & \textbf{description} & \textbf{example} \\
\midrule
\textbf{phonetics} & vocab's sound patterns & \emph{apparent} sounds like “a bare Asian.” \\
\addlinespace
\textbf{orthography} & written/spelling patterns & \emph{abet} looks like “a + bet.” \\
\addlinespace
\textbf{morphology} & structure including free and bound morphemes & \emph{aggrandize} = a + grand + –ize, to mean to make grander. \\
\addlinespace
\textbf{etymology} & vocab origin and history & \emph{adumbrate} comes from Latin ad- (to, on) + umbra (shade) + ate, to mean foreshadow or outline. \\
\addlinespace
\textbf{semantics} & vocab meaning and relationships & \emph{confound} has similar meaning and history with 'confuse'. \\
\bottomrule
\end{tabularx}
\end{table*}

\subsection{LLMs: linguistic competence and creativity}

\section{In-context learning performance} \label{sec:icl-performance}

%% TODO: Review in-context learning literature here, including CoT, few-shot prompting, and zero-shot prompting. Discuss the differences between these methods and their implications for LLMs' performance in generating mnemonics.
\subsection{Experimental setup}
We systematically compared various in-context learning approaches to understand how different prompting techniques affect mnemonic generation. \Cref{fig:prompting-methods} illustrates the percentage of linguistically grounded mnemonics generated by different prompt formulations.

We used \verb|curator| \citep{BespokeLabBESPOKE2025} with \verb|litellm| orchestration layer to interact with LLM APIs, simpify API calls, manage rate limits, and handle retries.

\subsection{Results}

% \begin{figure}
%   \centering
%   \includegraphics[width=\linewidth]{figures/prompt_comparison.png}
%   \caption{Comparison of prompting methods (see detailed prompt in \Cref{app:prompt-usage}). Y-axis shows percentages of linguistically-grounded mnemonics generated out of 50 requests sent for each prompt type.}
%   \label{fig:prompting-methods}
% % \end{figure}

We observed significant variation in the quality and linguistic grounding of generated mnemonics based solely on prompt formulation. Four distinct prompting strategies were evaluated (see details in \cref{app:prompt-usage})
Vanilla
Reasoning LLMs tend to overthink \citep{xuChainDraftThinking2025}
Good practices: provide decomposed instructions, structured output format, demonstration examples \citep{MishraREFRAMING2022}, and clarify definitions of linguistic features \citep{yinDidYouRead2023}

compress task definition \citep{yinDidYouRead2023},


\section{Knowledge and reasoning distillation} \label{sec:distillation}

%% \begin{figure}
%%   \centering
%%   \includegraphics[width=\linewidth]{figures/pipeline.pdf}
%%   \caption{\links pipeline. The pipeline consists of two main components: (1) CoT data generation and (2) model distillation. In the first step, we generate a dataset of mnemonics with reasoning traces using a large language model (LLM) as a teacher. In the second step, we distill the reasoning capabilities of the teacher model into a smaller student model using GRPO \Cref{app:grpo}. The final model can be deployed locally for vocabulary learning tasks.}
%%   \label{fig:distillation}
%% \end{figure}

We present \links, a pipeline that distills linguistic knowledge and reasoning through a teacher-student framework. This approach generates linguistically grounded mnemonics with reasoning traces and exemplifying sentences for vocabulary learning.

\subsection{Data construction} \label{sec:data-gen}
To generate high-quality linguistically grounded mnemonics, we first created a comprehensive training dataset. Following best practices in synthetic data generation with LLMs \citetext{\citealp{longLLMsDrivenSyntheticData2024b}, \citealp{openthoughtsteamOpenThoughts2025}}, we designed a data construction pipeline with several key components.

\paragraph{Vocabulary collection} We collected 5,000 distinct vocabulary words from four complementary sources: English as a foreign language tests (TOEFL iBT, IELTS Academic), standardized tests (SAT, GRE), CEFR levels C1 and C2 word lists, and the Oxford Dictionary of Philosophy. We selected these sources to ensure coverage of academic and abstract vocabulary that would benefit from mnemonic devices. After deduplication and fuzzy-matching decontamination, we refined our dataset to 2,000 distinct vocabulary words for post-training.

\paragraph{Prompt design} Based on the findings from \cref{sec:icl-performance}, we crafted system and user prompts that encouraged linguistically grounded reasoning. Our system prompt instructed the model to analyze potential linguistic features before generating a mnemonic. We used structured output format with designated sections for reasoning, mnemonic, and example, allowing for clearer evaluation and potential future extraction of specific components.

\paragraph{Teacher model selection} We selected \teachermodel (670B parameters) \citep{DeepSeek-AIDEEPSEEKR12025} for its advanced reasoning capabilities and extensive knowledge as the teacher model.

\paragraph{Dataset generation} Using the designed prompts and vocabulary list, we generated the \links, a dataset of approximately 2,000 entries, each containing: (1) a detailed reasoning trace exploring multiple linguistic features of the target vocabulary, (2) a concise mnemonic device leveraging the most salient linguistic connection, and, (3) an exemplifying sentence demonstrating proper usage.

\paragraph{Quality control} To ensure the quality of the generated mnemonics, we implemented a multi-step validation process. We first filtered out any entries that did not meet our structured output format or contained incomplete reasoning traces. We then performed a manual review of a random sample of 200 entries to assess the linguistic grounding and coherence of the mnemonics. This review process involved checking for clear connections between the vocabulary and the mnemonic, as well as ensuring that the example sentence accurately reflected the vocabulary's meaning.

\subsection{Training and inference}
After generating the synthetic dataset, we implemented a distillation process to transfer this linguistic reasoning capability to a smaller, easier-to-deploy model.

\paragraph{Student model selection} We selected \studentmodel as our student model due to its balance of performance, size, and deployment flexibility. This instruction-tuned variant of Google's Gemma-3 (1 billion parameters) \citep{GemmaTeamGEMMA2025} offers several advantages: (1) demonstrated instruction-following abilities, (2) multilingual capabilities for potential cross-lingual applications, and (3) compact size enabling deployment on consumer hardware including Apple Silicon.

\paragraph{Group Relative Policy Optimization (GRPO)} We employed GRPO \citep{DeepSeek-AIDEEPSEEKR12025} to distill the reasoning capabilities of the teacher model into the student model. The GRPO process consists of three main steps: (1) generating multiple candidate outputs for each input, (2) scoring these outputs using a reward model(s), and (3) updating the student model's policy based on the scores. GRPO technical details are included in \Cref{app:grpo}  and our used configuration is provided in \Cref{app:grpo}.

We defined three reward functions that encode basic characteristics of effective mnemonics:
(1) adherence to the structured format with reasoning, mnemonic, and example,
(2) explicit incorporation of linguistic features in \Cref{tab:linguistic-features}, and
(3) usage of the target vocabulary in the mnemonic,  penalizing bad mnemonics such as acronyms.
These reward functions operate directly on model outputs, assigning scalar scores based on how well the generation satisfies each criterion. The scores are then combined using weighted summation, with higher weights assigned to criteria 1 and 2. We generated two candidate outputs per training example to enable reinforcement from comparisons. Training was performed on a single NVIDIA H100 GPU approximately 4 hours.

\paragraph{Low-Rank Adaptation (LoRA)} We trained \studentmodel using GRPO wrapped in LoRA layers (\Cref{app:lora}) to reduce the number of trainable parameters and rank-stabilized LoRA that maintains stability for adapters with higher ranks. Full LoRA configuration is provided in \Cref{app:lora-config}.

\paragraph{Model quantization and serving} To enable efficient deployment on consumer hardware, we quantized the final model using 4-bit quantization with the NormalFloat (NF4) data type \citep{dettmersQLoRAEfficientFinetuning2023}. This significantly reduced the model size while maintaining performance quality. The quantized model can be served with a local application such as OpenWebUI (interface) and Ollama (command-line), making it accessible for language learners without requiring continuous internet connectivity or sharing potentially sensitive language learning data with third-party services.

\section{Evaluation}
\subsection{Experimental setup}

\paragraph{LLM-as-a-judge for 1-5 Likert ratings}

\paragraph{Pairwise preference using double-blind annotation}

\paragraph{Interactive side-by-side comparison}

\subsection{Results}

\section{Related work}
\section{Discussion}
\section{Conclusion}
\section{Limitations}

\section*{Acknowledgements}
% Bibliography entries for the entire Anthology, followed by custom entries
\bibliography{custom}

\clearpage


\begin{appendices}
  \section{Mnemonics: Linguistic Features and Characteristics} \label{app:mnemonics}


\subsection{Full mnemonic characteristiics} \label{app:mnemonic-characteristics}

\begin{itemize}
  \item Clear explanation linking the vocabulary to the mnemonic.
  \item Correct usage and definition of the vocabulary within the mnemonic.
  \item Strong association between the vocabulary and the mnemonic.
  \item Mnemonic is easy to understand, using similar or simpler vocabulary than the target term.
  \item Mnemonic is memorable, incorporating animate or concrete imagery, relevant contexts, or elements that evoke emotional responses.
\end{itemize}

\textbf{One of the following could make bad mnemonics}
\begin{itemize}
  \item Lack one of the three components.
  \item Incorrect definition or usage of the vocabulary.
  \item Circular association where the mnemonic simply repeats the vocabulary without adding meaning. Polysemous words (words with multiple meanings) are not considered circular.
  \item Weak or unclear association between mnemonic and vocabulary.
  \item Use semantically complex or obscure words that are more difficult than the target vocabulary.
  \item Mnemonic is abstract, making it hard to visualize or relate to.
  \item Use offensive or inappropriate language.
\end{itemize}

Unless specifically requested by users, LLMs are prompted to avoid culturally-specific mnemonics, such as those based on idioms or proverbs, as they may not be universally understood. This is particularly important for learners from diverse backgrounds and cultures.
\subsection{Linguistic features} \label{sec:linguistic-features}

\begin{table*}[!htb]
\centering
\caption{Examples of feature categories for English words.}
\label{tab:linguistic-features}
\begin{tabularx}{\textwidth}{l >{\raggedright\arraybackslash}X >{\raggedright\arraybackslash}X}
\toprule
\textbf{feature} & \textbf{description} & \textbf{example} \\
\midrule
\textbf{phonetics} & sound patterns & \emph{apparent} sounds like "a bare Asian." \\
\addlinespace
\textbf{orthography} & written/spelling patterns & \emph{abet} looks like "a + bet." \\
\addlinespace
\textbf{morphology} & modern English forms, including free and bound morphemes & \emph{aggrandize} = a + grand + –ize, to mean to make grander. \\
\addlinespace
\textbf{etymology} & origin and history & \emph{adumbrate} comes from Latin ad- (to, on) + umbra (shade) + ate, to mean foreshadow or outline. \\
\addlinespace
\textbf{semantics} & meaning and semantic relationships & \emph{confound} has similar meaning and history with 'confuse'. \\
\bottomrule
\end{tabularx}
\end{table*}

  
\section{Prompt usage} \label{app:prompt-usage}

All of the following prompts are system prompts for \teachermodel. We also added \Cref{app:mnemonic-characteristics} as mnemonic requirements for all prompts, and added the phrase "Let's think step by step" to the prompt variants sent to \xteachermodel. To conserve tokens, we remove unnecessary words that did not contribute to task instructions, such as modifiers.

\subsection*{Vanilla vs. Alternative Phrasing}

Vanilla prompt
\begin{quotation}
  A mnemonic to help learn and remember meaning of English vocabulary: \{term\}. Mnemonic should have following characteristics:

  [INSERT MNEMONIC REQUIREMENTS HERE].
\end{quotation}

Vanilla-Alternative prompt, with "linguistically grounded" added:
\begin{quotation}
  A \lgm
\end{quotation}

We observed that the vanilla prompts, with the term "mnemonic", often elicit backronyms (i.e. an existing word's letters are expanded into a phrase), initialisms or list. We hypothesized that this was because commonly encountered mnemonics are used for long information and non-linguistic contexts, and the training data reflects that popular use of mnemonic devices. This effect can be mitigated by adding the term "lingusitically grounded" to the prompt, to steer the model towards analyzing linguistic features, especially etymology and morphology.

\subsection*{Structured Output and Task Description}
We found improved performance with prompts that explicitly request broader linguistic analysis and provides structured output format with output descriptions \citep{MishraREFRAMING2022,yinDidYouRead2023}. We also tried to reduce the overthinking tendency in LLMs

\begin{quotation}
Generate a \lgm to help me learn and remember the meaning of English vocabulary: \{term\}.

Analyze linguistic features for this word (etymology, morphology, phonetics, orthography, semantics, etc). Stop linguistic analysis when you have a good linguistic connection. You must use that linguistic feature to form a mnemonic for the word.

Mnemonic should have following characteristics:
[INSERT MNEMONIC REQUIREMENTS HERE].

Provide output in this format:

- linguistic\_feature: chosen linguistic feature for mnemonic

- mnemonic: association + mnemonic

- example: example sentence of the vocabulary in context
\end{quotation}

This approach yielded a higher percentage of mnemonics with clear linguistic association, and better mnemonics overall. We also found that the model was more likely to use the same linguistic feature in the mnemonic as in the analysis, which is a key requirement for our task.

\subsection*{Added CoT examples}

For this prompt, we reused the task instructions and added 10 human-written CoT examples, each demonstrating the process of finding linguistic association of the vocabulary before constructing a mnemonic.

  
\section{Annotation details} \label{app:annotation}

For our double-blind annotation study, we followed best practices from \citet{tsengBestPracticesManaging2020} to ensure unbiased evaluation. The annotation process involved the following steps:

\numlist{1} We randomly selected 50 vocabulary terms from our test set, ensuring representation across different linguistic categories and vocabulary difficulty levels.

\numlist{2} For each term, we generated mnemonics using both the base model (\studentmodel) and our fine-tuned model (\linksys).

\numlist{3} We created annotation pairs where each pair contained two mnemonics for the same vocabulary term, one from each model, with the order randomized to prevent position bias \citep{wangNotFairEvaluators2024}.

\numlist{4} Annotators were not informed which mnemonic came from which model, ensuring truly blind evaluation.

\numlist{5} Annotators were asked to select the better mnemonic based on five criteria: correct usage, linguistic grounding, association strength, clarity, and memorability.

\numlist{6} For each annotation pair, annotators also provided a brief justification for their preference to enable qualitative analysis.

The two annotators were the author and \judgemodel. While this is a limitation of our study that we acknowledge in \Cref{sec:limitations}, we took several measures to reduce potential bias:

\numlist{1} The author completed annotations before running any quantitative analysis to prevent knowledge of statistical patterns from influencing judgment.

\numlist{2} \judgemodel was prompted to evaluate mnemonics according to the criteria in \Cref{fig:good-bad-mnemonics} without any information about which model generated which mnemonic.

\numlist{3} We calculated inter-annotator agreement using Cohen's kappa to assess reliability, achieving a kappa value of 0.72, indicating substantial agreement.

For future work, we plan to expand the annotation team to include language education experts and language learners to provide more diverse perspectives on mnemonic quality.

  
\appsection{Technical preliminaries} \label{app:technicality}

\appsubsection{In-context learning} \label{sec:icl-info}

\appsubsubsection{Chain-of-Thought (CoT) prompting} CoT \citep{weiChainofThoughtPromptingElicits2022} is a prompting technique that encourages LLMs to generate intermediate reasoning steps before arriving at a final answer. This approach has been shown to improve performance on complex tasks by guiding the model through a structured thought process.

\appsubsection{Neural Language Models and Transformer Architecture} \label{app:llm-transformer}

Neural language models are probabilistic frameworks that assign probabilities to sequences of words or subword units, known as tokens. A token is the smallest unit of text that the model processes, which can be as granular as individual characters, subwords, or entire words, depending on the tokenization strategy employed.

Given a sequence of tokens \( \mathbf{x} = (x_1, x_2, \ldots, x_T) \), a language model estimates the joint probability \( P(\mathbf{x}) \) by factorizing it into conditional probabilities:

\begin{equation}
P(\mathbf{x}) = \prod_{t=1}^T P(x_t \mid x_1, x_2, \ldots, x_{t-1})
\end{equation}

At each time step \( t \), the model predicts the next token \( x_t \) based on preceding sequence \( (x_1, x_2, \ldots, x_{t-1}) \). This autoregressive approach enables the generation of coherent text by sequentially predicting subsequent tokens.

The Transformer architecture underpins many state-of-the-art language models due to its efficiency and capability to model long-range dependencies. It utilizes self-attention mechanisms to weigh the relevance of each token in a sequence relative to others, regardless of their positions. The architecture comprises stacked layers, each including multi-head self-attention and position-wise feed-forward networks, facilitating parallelization and effective learning of complex patterns in data.

\subsubsection{Tokenizer} \label{app:tokenizer}

A tokenizer is a preprocessing tool that converts raw text into tokens, aligning the text with the LM's vocabulary. Tokenizers can employ various strategies, such as word-based, character-based, or subword-based tokenization, each with distinct advantages and use cases.

Byte Pair Encoding (BPE) is a subword tokenization algorithm that operates on the byte representation of text, enabling consistent handling of various scripts and special characters. It iteratively merges the most frequent pairs of adjacent bytes to form subword units, constructing a vocabulary that efficiently represents the training corpus. This method allows the tokenizer to decompose rare words into meaningful subword components, enhancing the model's capacity to process diverse and unseen terms.

For instance, the word "preposterous" might be tokenized into subwords like "pre", "poster", and "ous," facilitating the model's understanding and generation of these subwords in novel contexts. This subword granularity enables the model to generalize across morphologically complex words and out-of-vocabulary words, enhancing its robustness and vocabulary coverage. However, not all subwords are valid morphemes, which can limit the model's ability to capture morphological structure accurately. For instance, \texttt{tiktoken} (OpenAI's tokenizer)\footnote{\href{https://platform.openai.com/tokenizer}{https://platform.openai.com/tokenizer}} recognizes "ephemeral" as a single subword rather than three morphemes ("ept", "hemera", "-al"), because the affixes are not explicitly segmented, and 'epheremal' is a rare word so BPE better learns it as a single token.

\appsubsection{Family of Fine-Tuning Methods} \label{app:finetuning}
Fine-tuning is the process of adapting a pre-trained model to a specific task T or domain D by updating its parameters on a target dataset \(\mathcal{D}\). This process is crucial for leveraging pre-trained models' knowledge and enhancing their performance on downstream tasks.

There are several approaches to fine-tuning, which can be categorized by: 1. the availability of labeled data (supervised vs unsupervised fine-tuning), 2. the extent of parameter updates (full-parameter vs parameter-efficient fine-tuning), and 3. task. We focus on supervised fine-tuning, which involves minimizing a task-specific loss function over a labeled dataset.

\appsubsubsection{Supervised Fine-Tuning (SFT)}\label{app:sft}

SFT involves adapting a pre-trained model to a target task by minimizing a task-specific loss function over a labeled dataset. For a dataset \( \mathcal{D} = \{(\mathbf{x}^{(i)}, \mathbf{y}^{(i)})\}_{i=1}^N \), where \( \mathbf{x}^{(i)} \) is the input and \( \mathbf{y}^{(i)} \) is the target output, the objective is to minimize:

\begin{equation}
\mathcal{L} = \frac{1}{N} \sum_{i=1}^N \ell(f(\mathbf{x}^{(i)}; \theta), \mathbf{y}^{(i)})
\end{equation}

where \( f(\mathbf{x}; \theta) \) represents the model's output with parameters \( \theta \), and \( \ell \) is the loss function, typically cross-entropy loss.

\appsubsubsection{Instruction tuning} \label{app:instruction-tuning-it}

Instruction-tuning is a specialized form of SFT \Cref{app:sft} where models are trained on datasets comprising instruction-response pairs. This approach enables models to generalize across various tasks described by natural language instructions, enhancing their ability to follow diverse prompts. Formally, an instruction-tuning dataset consists of pairs \( \{(\mathbf{I}^{(i)}, \mathbf{y}^{(i)})\}_{i=1}^N \) or triplets \( \{(\mathbf{I}^{(i)}, \mathbf{x}^{(i)}, \mathbf{y}^{(i)})\}_{i=1}^N \), where \( \mathbf{I}^{(i)} \) denotes the instruction, \( \mathbf{x}^{(i)} \) is the optional input, and \( \mathbf{y}^{(i)} \) is the desired output. The training objective is to minimize the loss:

\begin{equation}
\mathcal{L} = \frac{1}{N} \sum_{i=1}^N \ell(f(\mathbf{I}^{(i)}, \mathbf{x}^{(i)}; \theta), \mathbf{y}^{(i)})
\end{equation}

where \( f \) represents the model parameterized by \( \theta \), and \( \ell \) is the loss function measuring the discrepancy between the model's prediction and the target output.

\appsubsubsection{Parameter-Efficient Fine-Tuning} \label{app:peft}
Full-parameter fine-tuning updates \textit{all} parameters of a pre-trained model on the target dataset, which can be computationally expensive and memory-intensive for large models. Parameter-efficient fine-tuning (PEFT) methods adjust only a subset of the parameters, reducing computational and storage requirements while maintaining performance \citep{XuPARAMETEREFFICIENT2023}.

The most common PEFT method is Low-Rank Adaptation (LoRA), and its variants. They are used in the training process as a wrapper around the model's weights, allowing for efficient updates without modifying the entire model. This approach is particularly useful for large models, where full fine-tuning may be impractical due to resource constraints.

\paragraph{Low-Rank Adaptation (LoRA)} decomposes the weight updates into low-rank matrices, reducing the number of trainable parameters \citep{huLoRALowRankAdaptation2021}. Specifically, for a weight matrix \( W \in \mathbb{R}^{d \times k} \), LoRA introduces two low-rank matrices \( A \in \mathbb{R}^{d \times r} \) and \( B \in \mathbb{R}^{r \times k} \), where \( 0 < r \ll \min(d, k) \). The adapted weight is:

\begin{equation}
W' = W + \alpha \cdot A B
\end{equation}

Here, \( \alpha \) is a scaling factor that controls the contribution of the low-rank adaptation. The rank \( r \) determines the capacity of the adaptation, balancing between expressiveness and efficiency.

LoRA introduces \( 2dr \) trainable parameters (size of \( A \) and \( B \)), which is significantly smaller than the original \( dk \) parameters. This reduction in parameters enables efficient fine-tuning of large models on limited hardware. In practice, LoRA is applied to specific modules of the model, such as attention and feed-forward layers, to balance performance and efficiency.

\paragraph{Rank-Stabilized LoRA (rsLoRA)} modifies the scaling factor in LoRA to improve performance across different ranks. The standard scaling factor \( \gamma_r = \alpha / r \) can slow learning for higher ranks. rsLoRA proposes adjusting the scaling factor to \( \gamma_r = \alpha / \sqrt{r} \), enhancing fine-tuning performance without increasing inference costs.

\subsection{Reinforcement Learning (RL)} \label{app:rl}

Reinforcement Learning (RL) is a framework in which an agent interacts with an environment to learn a policy $\pi_\theta$ that maximizes a long-term reward. At each time step $t$, the agent observes a state, takes an action, and receives a reward $r_t$. The goal is to maximize the expected cumulative reward, given by

\begin{equation}
J(\theta) = \mathbb{E}_{\pi_\theta}\left( \sum_{t=0}^{T} \gamma^t\,r_t\right)
\end{equation}

where $\gamma\in(0,1)$ is a discount factor.

In the next section, we consider an advanced method called Group Relative Policy Optimization (GRPO). GRPO extends PPO by generating multiple responses per prompt, comparing the rewards within each group, and adjusting the policy based on relative advantages. This online RL approach continuously improves by (1) generating completions, (2) scoring them using reward models, and (3) updating the model's policy with both an advantage term and a KL penalty.

\appsubsubsection{Group Relative Policy Optimization} \label{app:grpo}

Group Relative Policy Optimization (GRPO) \citet{DeepSeek-AIDEEPSEEKR12025} is an online reinforcement learning method specifically designed for scenarios where the model generates multiple responses (or completions) for the same prompt. It was introduced to improve the mathematical reasoning capabilities of LLMs, by generating multiple CoT responses for a given problem and then compares results to the ground truth.

Intuitively, GRPO generates multiple responses for a given prompt, scores them using reward models, calculates the relative reward of the group, and then compares each response's score to that relative reward to determine which is better or worse. The model then updates its policy to favor high-reward responses.

\paragraph{Generating completions} For each prompt $q$ in a batch, the model generates a set of $G$ completions:
\begin{equation}
O_q = \{o_1, o_2, \ldots, o_G\}
\end{equation}

Each completion $o_i$ consists of a sequence of tokens:
\begin{equation}
o_i = \{o_{i,1}, o_{i,2}, \ldots, o_{i,|o_i|}\}
\end{equation}

\paragraph{Computing the advantage} For each completion, a reward $r_i$ is computed using predefined reward functions. To enable comparison within groups, the rewards are normalized:
\begin{equation}
\mu_r = \text{mean}(r)
\end{equation}
\begin{equation}
\sigma_r = \text{std}(r)
\end{equation}
\begin{equation}
\hat{A}_{i,t} = \frac{r_i - \mu_r}{\sigma_r}
\end{equation}

where $r = \{r_1, r_2, \ldots, r_G\}$ is the set of rewards for all completions in the group, and $\hat{A}_{i,t}$ is the advantage for token $t$ in completion $i$. This normalization gives the method its name: Group Relative Policy Optimization.

\paragraph{Estimating the KL divergence} To prevent the policy from deviating too far from the reference policy $\pi_{\text{ref}}$, the KL divergence is estimated:
\begin{equation}
\pi_\text{ratio} = \frac{\pi_\theta(o_{i,t} | q, o_{i,<t})}{\pi_{\text{ref}}(o_{i,t} | q, o_{i,<t})}
\end{equation}
\begin{equation}
\pi_\text{inv\_ratio} = \frac{\pi_{\text{ref}}(o_{i,t} | q, o_{i,<t})}{\pi_\theta(o_{i,t} | q, o_{i,<t})}
\end{equation}
\begin{equation}
D_{\text{KL}} = \log\pi_\text{ratio} - 1 + \pi_\text{inv\_ratio}
\end{equation}

\paragraph{Computing the loss} The GRPO objective combines the advantage term with a KL penalty:
\begin{equation}
L_{\text{adv}} = -\frac{1}{G}\sum_{i=1}^{G}\sum_{t=1}^{|o_i|}\pi_\text{ratio}\hat{A}_{i,t}
\end{equation}
\begin{equation}
L_{\text{KL}} = \beta D_{\text{KL}}
\end{equation}
\begin{equation}
L_{\text{GRPO}}(\theta) = L_{\text{adv}} - L_{\text{KL}}
\end{equation}

where $\beta$ is a hyperparameter that controls the weight of the KL penalty. The advantage term encourages the policy to assign higher probability to tokens that lead to better rewards, while the KL term ensures that the policy doesn't deviate too far from the reference policy.

\paragraph{Multiple updates} For multiple $\mu$ updates after each generation, GRPO uses a clipped surrogate objective. First, compute the old policy ratio:
\begin{align}
\pi_{\text{old\_ratio}} &= \frac{\pi_\theta(o_{i,t} \mid q, o_{i,<t})}{\pi_{\theta_{\text{old}}}(o_{i,t} \mid q, o_{i,<t})},
\end{align}
then clip it:
\begin{align}
\pi_{\text{clipped}} &= \text{clip}\Bigl(\pi_{\text{old\_ratio}},\, 1-\epsilon,\, 1+\epsilon\Bigr).
\end{align}
The clipped advantage loss is $L_{\text{adv\_clipped}}$
\begin{equation}
-\frac{1}{G}\sum_{i=1}^{G}\sum_{t=1}^{|o_i|}
\min\Bigl(\pi_{\text{old\_ratio}}\hat{A}_{i,t},\, \pi_{\text{clipped}}\hat{A}_{i,t}\Bigr),
\end{equation}
yielding the final objective:
\begin{equation}
L_{\text{GRPO\_clipped}}(\theta) = L_{\text{adv\_clipped}} - L_{\text{KL}}.
\end{equation}

Here, $\epsilon$ (small constant, typically 0.2) controls how much the policy can change in a single update and $\beta$ controls the KL penalty's strength.

In HuggingFace's \texttt{trl} library, GRPO is implemented in the \texttt{GRPOTrainer} class and number of updates $\mu$ is controlled by the \texttt{num\_iterations} parameter. The default value of $\mu = 1$ simplifies the objective to the original GRPO formulation.

  \input{appendix/config}
  \appsection{Costs} \label{app:cost}


  
\section{Documentation of previous iterations} \label{app:previous-iterations}
\subsection{Fine-tune OpenAI} \label{app:openai-finetune}

\subsection{Fine-tune Gemma-3-4b-it} \label{app:gemma-finetune}

For supervised fine-tuning (SFT), we utilized the \texttt{trl} library with the following hyperparameters: batch size \( b = 16 \), number of epochs \( \text{eps} = 4 \), learning rate \( \alpha = 2 \times 10^{-5} \), weight decay \( \lambda = 0.05 \), and a cosine annealing learning rate scheduler with restarts.

The batch size \( b \) defines the number of training examples processed simultaneously during each forward and backward pass. A batch size of 16 balances computational efficiency and gradient estimation accuracy. Training for 4 epochs (\( \text{eps} = 4 \)) means the model will see the training data a total of four times, which ensures sufficient exposure to the training data without risking overfitting. The learning rate \( \alpha \) controls the step size for weight updates; a value of \( 2 \times 10^{-5} \) is typical for fine-tuning large language models, facilitating gradual convergence. Weight decay \( \lambda \) serves as a regularization term, penalizing large weights to prevent overfitting. The cosine annealing scheduler adjusts the learning rate following a cosine decay pattern, periodically restarting to allow the model to escape local minima and potentially achieve better generalization, compared to linear decay.

\section{Reflection} \label{sec:reflection}

  \section{Minerva Appendix: LOs \& HCs} \label{sec:minerva}

\subsection{LOs} \label{sec:los}
\numpara{CS110-codeReadability} The codebase exemplifies best practices in Python programming, adhering strictly to PEP conventions. Each module is documented with detailed Google-style docstrings and descriptive inline comments to ensure that the logic behind functions and classes is transparent to collaborators and future users. By utilizing tools like Ruff for linting and formatting, and mypy for type-checking, the codebase achieves a consistent style and minimizes errors. Additionally, the inclusion of pre-commit hooks ensures that these standards are maintained across all contributions, fostering a robust and maintainable codebase.
\numpara{CS162-communication} The documentation strives to adhere to industry standards, by including an informative README, clear commit messages (using conventional commit or gitmoji), and documented pull request. Each module has a module-level docstring and function-level docstrings, to ensure that the code is understandable to users and collaborators. I used \verb|Ruff| (for linting and black-style code formatting) and \verb|mypy| (for type checking), to enforce consistent and error-free code presentation, to facilitate readability and maintainability for the author and future collaborators (if any).
\numpara{CS156-MLCode} The machine learning pipeline was designed in Python to be both functional and comprehensible. The code integrates data processing, model fine-tuning, and hyperparameter tuning into a seamless pipeline, with each step working and documented. Model evaluation are explicitly defined in appendix \Cref{app:training-details}, ensuring replicability.
\numpara{CS156-MLExplanation} Currently, most details and explanations are defined in section \Cref{app:technicality} and appendix \Cref{app:training-details}. The final paper will provide more high-level diagrams of the machine learning techniques used in fine-tuning the Gemma-2 model. The supporting diagrams will visualize key processes, such as hyperparameter tuning and model evaluation. It will ensure that the methodology and results are accessible to a less specialized audience.
\numpara{CS162-separationofconcerns} The codebase is organized into distinct Python modules, each focused on a specific task such as data processing, mnemonic processing, and model fine-tuning. This separation of concerns aligns with best practices in software design, ensuring that each function is highly cohesive and performs a single well-defined responsibility. By maintaining modularity, the codebase facilitates easier debugging, testing, and future scaling, contributing to its long-term maintainability and effectiveness.

% TODO: Capstone LOs
\subsection{Capstone LOs} \label{sec:capstone-los}
\numpara{navigation} \Cref{app:previous-iterations}
\numpara{outcomeanalysis}
\numpara{curation}
\numpara{qualitydeliverables}

\subsection{HCs} \label{sec:hcs}
\numpara{audience} The project addresses dual audiences: (1) NLP researchers investigating LLMs' linguistic capabilities, for whom we provide detailed technical methodologies and evaluation metrics; and (2) educational technology developers seeking practical approaches to vocabulary learning assistance, for whom we demonstrate application potential and implementation strategies. For example, \Cref{sec:mnemonic-review} introduces key concepts about mnemonic devices and language teaching with sufficient depth for both audiences, while hiding technical details in appendices but providing pointers to those details in each section for interested readers. The paper's structure and content are designed to facilitate knowledge transfer across these domains, ensuring that both audiences can derive value from the research findings. Technical content is balanced with practical implications for vocabulary acquisition and language learning, ensuring relevance to both research and application-oriented readers.

\numpara{organization} The paper follows standard ACL formatting conventions, with a logical progression from theoretical foundations through methodology to empirical results. The structure facilitates efficient information extraction, with each section building upon previous content. Key contributions are identified early (\Cref{sec:intro}) and systematically developed throughout subsequent sections. Technical details that might interrupt argumentative flow are relegated to appendices, maintaining narrative coherence while ensuring methodological transparency for reproduction purposes. This organization aligns with expectations of the computational linguistics community while supporting efficient knowledge transfer.

\numpara{gapanalysis} \Cref{sec:intro} identifies critical limitations in existing mnemonic generation approaches: (1) overreliance on the keyword method, which fails for abstract vocabulary lacking concrete referents, and (2) neglect of the rich linguistic knowledge embedded in LLMs that could enable more diverse mnemonic strategies beyond simple keyword associations. Prior work has also passively delivered mnemonics to learners rather than leveraging individual learning preferences, despite research showing self-created mnemonics enhance retention. These identified gaps motivate our project to elicit linguistic reasoning and creativity from LLMs, enabling them to generate mnemonics that are not only effective but also linguistically grounded. Our final model, \linksys, after deployment, could interact with learners to tailor mnemonic generation based on their preferences, enhancing the learning experience. This approach addresses the limitations of existing methods by providing a more comprehensive and personalized vocabulary learning tool. The project also contributes to the field of educational technology by exploring how LLMs can be harnessed for creating effective resources for language education and self-study, such as creating mnemonic devices.

% TODO
\numpara{hypothesisdevelopment} The research questions in \Cref{sec:intro} establish testable hypotheses regarding LLMs as linguistic knowledge bases for mnemonic generation. We hypothesize that fine-tuning LLMs on linguistically annotated examples will improve mnemonic quality across semantic relevance, diversity, and helpfulness dimensions. This hypothesis is predicated on the theoretical assumption that LLMs encode significant linguistic knowledge during pre-training that can be accessed through targeted fine-tuning. The hypothesis is operationalized through specific evaluation metrics described in \Cref{sec:evaluation}, ensuring empirical testability.

\numpara{scienceoflearning}

\numpara{optimization} ...

\numpara{algorithms} The fine-tuning methodology detailed in \Cref{app:finetuning} incorporates algorithmic innovations in parameter-efficient adaptation. Specifically, we implement QLoRA with rank-stabilized scaling (rsLoRA), modifying the standard scaling factor to improve performance across different ranks. This algorithm reduces memory requirements by quantizing the pre-trained model to 4-bit precision while allowing selective updates to low-rank adaptation matrices. Population-based training explores the hyperparameter space dynamically, pruning underperforming configurations and exploring promising regions to optimize validation performance.

\numpara{heuristics} Our approach employs several problem-solving heuristics to enhance mnemonic generation. The linguistic feature classification system provides a structured framework for identifying and leveraging different linguistic aspects in vocabulary. We use problem decomposition by separating mnemonic creation into linguistic analysis and creative association phases, enabling more systematic knowledge utilization. Visualization techniques in the form of embedding space projections help identify semantic relationships between vocabulary terms and potential mnemonic content. These heuristics guide both the model fine-tuning process and the subsequent evaluation methodology.

\numpara{sampling} The evaluation methodology employs stratified random sampling to ensure representation across linguistic feature categories (\Cref{sec:vocab-selection}) and vocabulary complexity levels. For human evaluation, we randomly selected 50 test examples, stratified by linguistic feature, to obtain less biased assessments of mnemonic quality. This sampling strategy ensures balanced representation of different mnemonic types while maintaining statistical validity. Each example received multiple independent ratings to mitigate individual rater bias, with inter-rater reliability calculated using Cohen's Kappa score to confirm consistency in ratings.

\numpara{dataviz} We employ targeted data visualizations to communicate complex relationships between linguistic features and mnemonic effectiveness. Radar charts display the distribution of linguistic features across different model outputs, while heatmaps visualize correlation patterns between computational metrics and human evaluations. Embedding space projections illustrate semantic relationships between vocabulary terms and their mnemonics, providing intuitive visual confirmation of semantic relevance scores. These visualizations enhance interpretability of results while supporting our conclusions regarding the contribution of different linguistic features to mnemonic quality.

\numpara{significance} Statistical significance testing (\Cref{sec:qualitative-llm-judge}) confirms the reliability of our comparative results between baseline and \linksys. We employ paired statistical tests (Wilcoxon signed-rank and McNemar test) to account for vocabulary-specific variation when comparing model outputs on identical test sets. Effect size calculations quantify the practical significance of improvements, while confidence intervals provide transparency regarding the precision of our estimates. Multiple comparison corrections maintain statistical rigor when evaluating performance across different linguistic feature categories.

\numpara{shapingbehavior} The intended application of our approach shapes vocabulary learning behavior by encouraging deeper engagement with linguistic features. Rather than keyword method, the generated mnemonics prompt learners to recognize morphological, etymological, and semantic patterns, fostering more robust mental representations. By explicitly highlighting these linguistic features, the system promotes analytical processing of vocabulary, which research indicates enhances long-term retention. This behavior-shaping aspect represents a significant advantage over keyword-only approaches that rely on shallow phonetic or orthographic associations.

\numpara{biasmitigation}

\numpara{ethicalconsiderations} Our work addresses ethical considerations in educational technology deployment. We explore linguistic diversity by analyzing and leveraging several linguistic features for mnemonic generation, ensuring that the generated mnemonics are inclusive and accessible to learners from diverse linguistic backgrounds. The model is trained on a diverse dataset of vocabulary words, including those from various languages and etymological origins, to ensure that the mnemonics generated are relevant and culturally sensitive. We also consider the potential for bias in the generated mnemonics by ensuring that the training data is representative of a wide range of vocabulary words.

\numpara{studyreplication} To facilitate replication, we provide comprehensive implementation details in \Cref{app:training-details}, including environment setup, model parameters, and training configurations. The dataset construction process is documented in \Cref{sec:data-gen}, with preprocessing steps explicitly specified. All code and datasets are made publicly available through GitHub and HuggingFace repositories, with standardized formats ensuring compatibility with common ML frameworks. This transparency ensures that other researchers can validate the findings and build upon the methodology.

\end{appendices}

\end{document}
