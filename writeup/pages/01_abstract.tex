\begin{abstract}

To acquire advanced vocabulary (CEFR B2+), English learners often use mnemonic devices, memorable associations linking a new concept to learned concepts to improve memory and recall. This paper introduces a novel approach to vocabulary learning by leveraging Large Language Models (LLMs) to generate \lgms (LGMs), memory aids that utilize a word's linguistic semantic features. We develop a taxonomy of linguistic features and define characteristics of effective mnemonics called Vocabulary-Association-Mnemonic (VAM) model. We investigated techniques for prompting large language models to \numlist{1} generate high-quality LGMs and \numlist{2} distill linguistic reasoning capabilities from a teacher model \teachermodel to the \textit{student model}, \studentmodel\footnote{\url{https://huggingface.co/collections/google/gemma-3-release-67c6c6f89c4f76621268bb6d}} to create \linksys, a specialized smaller model for mnemonic generation, using the base model \studentmodel, GRPO, and \links, a synthetic dataset of 2000 triplets of \textit{reasoning trace, mnemonic, and example sentence} for 2000 vocabulary useful for TOEFL iBT \footnote{Internet-based Test of English as a Foreign Language}, IELTS Academic \footnote{International English Language Testing System}, and SAT\footnote{Scholastic Aptitude Test}. Through both LLM-as-a-judge and preference evaluation, we demonstrate that \linksys produces mnemonics that significantly outperform those from base model,  across multiple quality metrics. Our results show that LGMs created by our system are not only preferred by learners but also provide deeper linguistic insights that potentially enhance vocabulary retention and understanding. The trained, quantized model \linksys can be served with local applications such as OpenWebUI (interface) and Ollama (command-line). This work examplifies that carefully designed NLP systems can generate resources for language learning, for both classroom settings and self-study.\footnote{\url{https://github.com/chiffonng/mnemonic-gen}}

\end{abstract}
