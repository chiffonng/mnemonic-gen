\begin{abstract} \label{sec:abstract}
To acquire advanced vocabulary, English learners often use mnemonic devices, memorable associations linking a new concept to learned concepts to improve memory and recall. Reviewing the literature on mnemonic techniques, we characterize good mnemonics as \textbf{linguistically grounded}, which better link to the target vocabulary, improving long-term retention and linguistic knowledge, especially at advanced levels (CEFR B2+). We investigate whether Large Language Models can consistently help write such effective mnemonics, with three different settings: in-context learning, and reasoning distillation. Concretely, we first measured different prompting strategies with a frontier reasoning model, \teachermodel, and generated \links, a synthetic dataset of 2000 triplets of \textit{reasoning trace, mnemonic, and example sentence} for 2000 vocabulary useful for TOEFL iBT \footnote{Internet-based Test of English as a Foreign Language}, IELTS Academic \footnote{International English Language Testing System}, and SAT\footnote{Scholastic Aptitude Test}. Second, using a subset of \links, we distilled linguistic reasoning from the \textit{teacher model} to the \textit{student model}, \studentmodel\footnote{\url{https://huggingface.co/collections/google/gemma-3-release-67c6c6f89c4f76621268bb6d}} , with online reinforcement learning. The trained, quantized model can be served with a local application such as OpenWebUI (interface) and Ollama (command-line).

Preliminary evaluation shows

The project examplifies that carefully designed NLP systems can generate resources for language learning, either in classroom settings or in self-study.\footnote{\url{https://github.com/chiffonng/mnemonic-gen}}
\end{abstract}
