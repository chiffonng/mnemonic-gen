
\section{Introduction} \label{sec:intro}
Vocabulary acquisition challenges many English language learners, particularly at upper intermediate to advanced levels where abstract and academic vocabulary predominates. Mnemonics, cognitive tools that help learners create associations between new vocabulary and familiar concepts, serve as valuable tools for enhancing retention and recall. The deeper learners elaborate the connection between the mnemonic and the target vocabulary, the longer and better they can recall the term. However, creating such effective mnemonics demands both linguistic expertise and creative effort, presenting a significant barrier for most learners. Large Language Models (LLMs) have demonstrated capabilities as knowledge bases and creative text generators, suggesting their potential for automated mnemonic generation.

Previous work explored automated mnemonic generation through computational methods using the \textbf{keyword method}, which involves \numlist{1} generateing simpler keywords that together sound or look like the target vocabulary and \numlist{2} creating memorable explanations that include the vocabulary, the keywords, amd its meaning \citep{atkinsonApplicationMnemonicKeyword1975}. \citet{savvaTransPhonerAutomatedMnemonic2014} and \citet{OzbalAUTOMATION2014} generated keywords of phonetic and orthographic similarities in the native language for foreign language vocabulary, across multiple languages. \citet{LeeSMARTPHONE2023} extended this work and utilized LLMs to produce phonetically similar keywords and visual cues and \citet{LeeEXPLORING2024} prompted LLMs to generate multiple mnemonic candidates and evaluate them based on imageability and coherence. Most recently, \citet{BalepurSMART2024} fine-tuned LLaMA-2-70B on 800 crowd-sourced mnemonics and aligned outputs with learner preferences and learning outcomes.

%%%% Figure Highlight the difference between keyword "mnemonic" and "linguistically grounded mnemonics" with an example
%%% (e.g., \textbf{preposterous} can be broken down as pre- (before) + poster (after) + ous. Anything that comes both before and after is preposterous)

Although the keyword method is commonly used and empirically validated in classroom and laboratory contexts \citetext{\citealp{atkinsonApplicationMnemonicKeyword1975}, \citealp{pressleyMnemonicKeywordMethod1982}}, it may lead to longer retrieval time \citep{vanhellKeywordMnemonicsRote1997} and be inadequate for fairly abstract vocabulary \citetext{\citealp{CamposLIMITATIONS2003}, \citealp{CamposUSING2011}}. Such methods typically neglect other rich linguistic knowledge embedded in LLMs that could provide diverse mnemonic strategies beyond simple keyword associations. Second, previous works passively deliver generated mnemonics to learners. While \textsc{Smart} \citep{BalepurSMART2024} was further trained on learners' preferences, these preferences were aggregated, potentially missing alignment with individual learning styles. Language learners who use self-created mnemonics retain vocabulary more effectively and for longer duration \citep{madanExploringWordMemorability2021}.

%% TODO: Add research questions, or some bridge to the next section

Our contributions can be summarized as follows. \numlist{1} We demonstrate that LLMs can generate \textbf{linguistically grounded mnemonics} (\Cref{sec:mnemonic-review}), which emphasizes the importance of linguistic features in creating effective mnemonics in a broader way than keyword method, through reasoning and creative writing (\Cref{sec:icl-performance}). \numlist{2} We present \links, a synthetic dataset of 2000 triplets of \textit{reasoning trace, mnemonic, and example sentence} for 2000 vocabulary. They can be integrated in a spaced repetition system (SRS) or language learning applications for vocabulary acquisition with better retrieval. \numlist{3} We distill the reasoning capabilities of a reasoning LLM like \teachermodel into a smaller model, \studentmodel (\Cref{sec:distillation}). The trained model, \linksys, can be served locally, enabling interested users to generate mnemonics without relying on external APIs.
