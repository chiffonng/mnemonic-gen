\section{Minerva Appendix: LOs \& HCs} \label{sec:minerva}

\subsection{LOs} \label{sec:los}
\numpara{CS110-codeReadability} The codebase exemplifies best practices in Python programming, adhering strictly to PEP conventions. Each module is documented with detailed Google-style docstrings and descriptive inline comments to ensure that the logic behind functions and classes is transparent to collaborators and future users. By utilizing tools like Ruff for linting and formatting, and mypy for type-checking, the codebase achieves a consistent style and minimizes errors. Additionally, the inclusion of pre-commit hooks ensures that these standards are maintained across all contributions, fostering a robust and maintainable codebase.
\numpara{CS162-communication} The documentation for strives to adhere to industry standards, by including an informative README, clear issue tracking, and documented pull request. Each module has a module-level docstring and function/class-level docstrings, to ensure that the code is understandable to users and collaborators. Linting and formatting tools, including Ruff and mypy, are employed to enforce consistent and error-free code presentation, to facilitate readability and maintainability for the author and future collaborators (if any).
\numpara{CS156-MLCode} The machine learning pipeline was designed in Python to be both functional and comprehensible. The code integrates data processing, model fine-tuning, and hyperparameter tuning into a seamless pipeline, with each step working and documented. Model evaluation are explicitly defined in appendix \Cref{app:training-details}, ensuring replicability.
\numpara{CS156-MLExplanation} Currently, most details and explanations are defined in section \Cref{app:technicality} and appendix \Cref{app:training-details}. The final paper will provide more high-level diagrams of the machine learning techniques used in fine-tuning the Gemma-2 model. The supporting diagrams will visualize key processes, such as hyperparameter tuning and model evaluation. It will ensure that the methodology and results are accessible to a less specialized audience.
\numpara{CS162-separationofconcerns} The codebase is organized into distinct Python modules, each focused on a specific task such as data processing, mnemonic processing, and model fine-tuning. This separation of concerns aligns with best practices in software design, ensuring that each function is highly cohesive and performs a single well-defined responsibility. By maintaining modularity, the codebase facilitates easier debugging, testing, and future scaling, contributing to its long-term maintainability and effectiveness

% TODO: Capstone LOs
\subsection{Capstone LOs} \label{sec:capstone-los}
\numpara{navigation} \Cref{app:previous-iterations}
\numpara{outcomeanalysis}
\numpara{curation}
\numpara{qualitydeliverables}

\subsection{HCs} \label{sec:hcs}
\numpara{gapanalysis} \Cref{sec:intro} identifies critical limitations in existing mnemonic generation approaches: (1) overreliance on the keyword method, which fails for abstract vocabulary lacking concrete referents, and (2) neglect of the rich linguistic knowledge embedded in LLMs that could enable more diverse mnemonic strategies beyond simple keyword associations. Prior work has also passively delivered mnemonics to learners rather than leveraging individual learning preferences, despite research showing self-created mnemonics enhance retention. These identified gaps motivate our development of a linguistic knowledge mining approach from LLMs.

\numpara{hypothesisdevelopment} The research questions in \Cref{sec:intro} establish testable hypotheses regarding LLMs as linguistic knowledge bases for mnemonic generation. We hypothesize that fine-tuning LLMs on linguistically annotated examples will improve mnemonic quality across semantic relevance, diversity, and helpfulness dimensions. This hypothesis is predicated on the theoretical assumption that LLMs encode significant linguistic knowledge during pre-training that can be accessed through targeted fine-tuning. The hypothesis is operationalized through specific evaluation metrics described in \Cref{sec:evaluation}, ensuring empirical testability.

\numpara{optimization} Our approach optimizes model performance through systematic hyperparameter tuning with the objective function minimizing validation loss. We employ parameter-efficient fine-tuning techniques (QLoRA with rsLoRA scaling) that significantly reduce computational requirements while maintaining performance. The hyperparameter space exploration includes learning rate, batch size, LoRA rank, and scaling factors, with population-based training enabling efficient identification of optimal configurations. This iterative optimization process is crucial for extracting maximal linguistic knowledge from the models while preventing overfitting to the training data.

\numpara{audience} The project addresses dual audiences: (1) NLP researchers investigating LLMs' linguistic capabilities, for whom we provide detailed technical methodologies and evaluation metrics; and (2) educational technology developers seeking practical approaches to vocabulary learning assistance, for whom we demonstrate application potential and implementation strategies. The Background section introduces key concepts with sufficient depth for computational linguists while remaining accessible to educational technology practitioners. Technical content is balanced with practical implications for vocabulary acquisition, ensuring relevance to both research and application-oriented readers.

\numpara{organization} The paper follows standard ACL formatting conventions, with a logical progression from theoretical foundations through methodology to empirical results. The structure facilitates efficient information extraction, with each section building upon previous content. Key contributions are identified early (\Cref{sec:intro}) and systematically developed throughout subsequent sections. Technical details that might interrupt argumentative flow are relegated to appendices, maintaining narrative coherence while ensuring methodological transparency for reproduction purposes. This organization aligns with expectations of the computational linguistics community while supporting efficient knowledge transfer.

\numpara{algorithms} The fine-tuning methodology detailed in \Cref{app:finetuning} incorporates algorithmic innovations in parameter-efficient adaptation. Specifically, we implement QLoRA with rank-stabilized scaling (rsLoRA), modifying the standard scaling factor to improve performance across different ranks. This algorithm reduces memory requirements by quantizing the pre-trained model to 4-bit precision while allowing selective updates to low-rank adaptation matrices. Population-based training explores the hyperparameter space dynamically, pruning underperforming configurations and exploring promising regions to optimize validation performance.

\numpara{heuristics} Our approach employs several problem-solving heuristics to enhance mnemonic generation. The linguistic feature classification system provides a structured framework for identifying and leveraging different linguistic aspects in vocabulary. We use problem decomposition by separating mnemonic creation into linguistic analysis and creative association phases, enabling more systematic knowledge utilization. Visualization techniques in the form of embedding space projections help identify semantic relationships between vocabulary terms and potential mnemonic content. These heuristics guide both the model fine-tuning process and the subsequent evaluation methodology.

\numpara{sampling} The evaluation methodology employs stratified random sampling to ensure representation across linguistic feature categories and vocabulary complexity levels. For human evaluation, we randomly selected 50 test examples, stratified by linguistic feature type, to obtain unbiased assessments of mnemonic quality. This sampling strategy ensures balanced representation of different mnemonic types while maintaining statistical validity. Each example received multiple independent ratings to mitigate individual rater bias, with inter-rater reliability calculated using Gwet's AC2 to confirm assessment consistency.

\numpara{studyreplication} To facilitate replication, we provide comprehensive implementation details in Appendix \Cref{app:training-details}, including environment configurations, model parameters, and hyperparameter settings. The dataset construction process is documented in \Cref{sec:data-gen}, with preprocessing steps explicitly specified. All code and datasets are made publicly available through GitHub and HuggingFace repositories, with standardized formats ensuring compatibility with common ML frameworks. This transparency ensures that other researchers can validate our findings and build upon our methodology.

\numpara{observationalstudy} Our human evaluation component constitutes an observational study assessing how learners perceive and utilize generated mnemonics. The Likert-scale ratings provide quantitative metrics, while qualitative feedback offers insights into specific features that enhance mnemonic effectiveness. Though limited in scale, this observational component validates computational metrics and provides preliminary evidence regarding learning impact. Future work will extend this observational approach through longitudinal studies tracking actual vocabulary retention over time.

\numpara{dataviz} We employ targeted data visualizations to communicate complex relationships between linguistic features and mnemonic effectiveness. Radar charts display the distribution of linguistic features across different model outputs, while heatmaps visualize correlation patterns between computational metrics and human evaluations. Embedding space projections illustrate semantic relationships between vocabulary terms and their mnemonics, providing intuitive visual confirmation of semantic relevance scores. These visualizations enhance interpretability of results while supporting our conclusions regarding the contribution of different linguistic features to mnemonic quality.

\numpara{significance} Statistical significance testing confirms the reliability of our comparative results between baseline and fine-tuned models. We employ paired statistical tests (Wilcoxon signed-rank) to account for vocabulary-specific variation when comparing model outputs on identical test sets. Effect size calculations quantify the practical significance of improvements, while confidence intervals provide transparency regarding the precision of our estimates. Multiple comparison corrections maintain statistical rigor when evaluating performance across different linguistic feature categories.

\numpara{shapingbehavior} The intended application of our approach shapes vocabulary learning behavior by encouraging deeper engagement with linguistic features. Rather than passive memorization, the generated mnemonics prompt learners to recognize morphological patterns, etymological connections, and semantic relationships, fostering more robust mental representations. By explicitly highlighting these linguistic features, the system promotes analytical processing of vocabulary, which research indicates enhances long-term retention. This behavior-shaping aspect represents a significant advantage over keyword-only approaches that rely on shallow phonetic associations.

\numpara{ethicalconsiderations} Our work addresses ethical considerations in educational technology deployment. We prioritize linguistic diversity by including vocabulary from various etymological backgrounds rather than focusing exclusively on Latin/Greek derivatives. The evaluation process incorporates feedback from learners with diverse linguistic profiles to ensure the approach benefits various learning styles. We acknowledge limitations regarding potential cultural specificity in some mnemonics and identify this as an area for future refinement. All human evaluation participants provided informed consent, and data collection followed established ethical guidelines for educational research.
