
\section{Prompt usage} \label{app:prompt-usage}

All of the following prompts are system prompts for \teachermodel. We also added \Cref{app:mnemonic-characteristics} as mnemonic requirements for all prompts, and added the phrase "Let's think step by step" to the prompt variants sent to \xteachermodel. To conserve tokens, we remove unnecessary words that did not contribute to task instructions, such as modifiers.

\subsection*{Vanilla vs. Alternative Phrasing}

Vanilla prompt
\begin{quotation}
  A mnemonic to help learn and remember meaning of English vocabulary: \{term\}. Mnemonic should have following characteristics:

  [INSERT MNEMONIC REQUIREMENTS HERE].
\end{quotation}

Vanilla-Alternative prompt, with "linguistically grounded" added:
\begin{quotation}
  A \lgm
\end{quotation}

We observed that the vanilla prompts, with the term "mnemonic", often elicit backronyms (i.e. an existing word's letters are expanded into a phrase), initialisms or list. We hypothesized that this was because commonly encountered mnemonics are used for long information and non-linguistic contexts, and the training data reflects that popular use of mnemonic devices. This effect can be mitigated by adding the term "lingusitically grounded" to the prompt, to steer the model towards analyzing linguistic features, especially etymology and morphology.

\subsection*{Structured Output and Task Description}
We found improved performance with prompts that explicitly request broader linguistic analysis and provides structured output format with output descriptions \citep{MishraREFRAMING2022,yinDidYouRead2023}. We also tried to reduce the overthinking tendency in LLMs

\begin{quotation}
Generate a \lgm to help me learn and remember the meaning of English vocabulary: \{term\}.

Analyze linguistic features for this word (etymology, morphology, phonetics, orthography, semantics, etc). Stop linguistic analysis when you have a good linguistic connection. You must use that linguistic feature to form a mnemonic for the word.

Mnemonic should have following characteristics:
[INSERT MNEMONIC REQUIREMENTS HERE].

Provide output in this format:

- linguistic\_feature: chosen linguistic feature for mnemonic

- mnemonic: association + mnemonic

- example: example sentence of the vocabulary in context
\end{quotation}

This approach yielded a higher percentage of mnemonics with clear linguistic association, and better mnemonics overall. We also found that the model was more likely to use the same linguistic feature in the mnemonic as in the analysis, which is a key requirement for our task.

\subsection*{Added CoT examples}

For this prompt, we reused the task instructions and added 10 human-written CoT examples, each demonstrating the process of finding linguistic association of the vocabulary before constructing a mnemonic.
