\section{Mnemonics: Linguistic Features and Characteristics} \label{app:mnemonics}


\subsection{Full mnemonic characteristiics} \label{app:mnemonic-characteristics}

\begin{itemize}
  \item Clear explanation linking the vocabulary to the mnemonic.
  \item Correct usage and definition of the vocabulary within the mnemonic.
  \item Strong association between the vocabulary and the mnemonic.
  \item Mnemonic is easy to understand, using similar or simpler vocabulary than the target term.
  \item Mnemonic is memorable, incorporating animate or concrete imagery, relevant contexts, or elements that evoke emotional responses.
\end{itemize}

\textbf{One of the following could make bad mnemonics}
\begin{itemize}
  \item Lack one of the three components.
  \item Incorrect definition or usage of the vocabulary.
  \item Circular association where the mnemonic simply repeats the vocabulary without adding meaning. Polysemous words (words with multiple meanings) are not considered circular.
  \item Weak or unclear association between mnemonic and vocabulary.
  \item Use semantically complex or obscure words that are more difficult than the target vocabulary.
  \item Mnemonic is abstract, making it hard to visualize or relate to.
  \item Use offensive or inappropriate language.
\end{itemize}

Unless specifically requested by users, LLMs are prompted to avoid culturally-specific mnemonics, such as those based on idioms or proverbs, as they may not be universally understood. This is particularly important for learners from diverse backgrounds and cultures.
\subsection{Linguistic features} \label{sec:linguistic-features}

\begin{table*}[!htb]
\centering
\caption{Examples of feature categories for English words.}
\label{tab:linguistic-features}
\begin{tabularx}{\textwidth}{l >{\raggedright\arraybackslash}X >{\raggedright\arraybackslash}X}
\toprule
\textbf{feature} & \textbf{description} & \textbf{example} \\
\midrule
\textbf{phonetics} & sound patterns & \emph{apparent} sounds like "a bare Asian." \\
\addlinespace
\textbf{orthography} & written/spelling patterns & \emph{abet} looks like "a + bet." \\
\addlinespace
\textbf{morphology} & modern English forms, including free and bound morphemes & \emph{aggrandize} = a + grand + –ize, to mean to make grander. \\
\addlinespace
\textbf{etymology} & origin and history & \emph{adumbrate} comes from Latin ad- (to, on) + umbra (shade) + ate, to mean foreshadow or outline. \\
\addlinespace
\textbf{semantics} & meaning and semantic relationships & \emph{confound} has similar meaning and history with 'confuse'. \\
\bottomrule
\end{tabularx}
\end{table*}
