\section{Linguistic Features of Mnemonics}
\label{sec:linguistic-features}

\begin{table*}[!htb]
\centering
\caption{Examples of feature categories for English words.}
\label{tab:linguistic-features}
\begin{tabularx}{\textwidth}{l >{\raggedright\arraybackslash}X >{\raggedright\arraybackslash}X}
\toprule
\textbf{feature} & \textbf{description} & \textbf{example} \\
\midrule
\textbf{phonetics} & sound patterns & \emph{apparent} sounds like "a bare Asian." \\
\addlinespace
\textbf{orthography} & written/spelling patterns & \emph{abet} looks like "a + bet." \\
\addlinespace
\textbf{morphology} & modern English forms, including free and bound morphemes & \emph{aggrandize} = a + grand + –ize, to mean to make grander. \\
\addlinespace
\textbf{etymology} & origin and history & \emph{adumbrate} comes from Latin ad- (to, on) + umbra (shade) + ate, to mean foreshadow or outline. \\
\addlinespace
\textbf{semantics} & meaning and semantic relationships & \emph{confound} has similar meaning and history with 'confuse'. \\
\bottomrule
\end{tabularx}
\end{table*}
