
\section{Annotation details} \label{app:annotation}

For our double-blind annotation study, we followed best practices from \citet{tsengBestPracticesManaging2020} to ensure unbiased evaluation. The annotation process involved the following steps:

\numlist{1} We randomly selected 50 vocabulary terms from our test set, ensuring representation across different linguistic categories and vocabulary difficulty levels.

\numlist{2} For each term, we generated mnemonics using both the base model (\studentmodel) and our fine-tuned model (\linksys).

\numlist{3} We created annotation pairs where each pair contained two mnemonics for the same vocabulary term, one from each model, with the order randomized to prevent position bias \citep{wangNotFairEvaluators2024}.

\numlist{4} Annotators were not informed which mnemonic came from which model, ensuring truly blind evaluation.

\numlist{5} Annotators were asked to select the better mnemonic based on five criteria: correct usage, linguistic grounding, association strength, clarity, and memorability.

\numlist{6} For each annotation pair, annotators also provided a brief justification for their preference to enable qualitative analysis.

The two annotators were the author and \judgemodel. While this is a limitation of our study that we acknowledge in \Cref{sec:limitations}, we took several measures to reduce potential bias:

\numlist{1} The author completed annotations before running any quantitative analysis to prevent knowledge of statistical patterns from influencing judgment.

\numlist{2} \judgemodel was prompted to evaluate mnemonics according to the criteria in \Cref{fig:good-bad-mnemonics} without any information about which model generated which mnemonic.

\numlist{3} We calculated inter-annotator agreement using Cohen's kappa to assess reliability, achieving a kappa value of 0.72, indicating substantial agreement.

For future work, we plan to expand the annotation team to include language education experts and language learners to provide more diverse perspectives on mnemonic quality.
