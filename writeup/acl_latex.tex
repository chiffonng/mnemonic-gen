\pdfoutput=1
% In particular, the hyperref package requires pdfLaTeX in order to break URLs across lines.

\documentclass[11pt]{article}

% Change "review" to "final" to generate the final (sometimes called camera-ready) version.
% Change to "preprint" to generate a non-anonymous version with page numbers.
\usepackage[preprint]{acl}

% Standard package includes
\usepackage{times}
\usepackage{latexsym}

% For proper rendering and hyphenation of words containing Latin characters (including in bib files)
\usepackage[T1]{fontenc}
% For Vietnamese characters
% \usepackage[T5]{fontenc}
% See https://www.latex-project.org/help/documentation/encguide.pdf for other character sets

% This assumes your files are encoded as UTF8
\usepackage[utf8]{inputenc}

% This is not strictly necessary, and may be commented out,
% but it will improve the layout of the manuscript,
% and will typically save some space.
\usepackage{microtype}

% This is also not strictly necessary, and may be commented out.
% However, it will improve the aesthetics of text in
% the typewriter font.
\usepackage{inconsolata}
\usepackage{amsmath}
\usepackage{amsfonts}
\usepackage{graphicx}
\usepackage{float}
\usepackage{tabularx} % For flexible-width columns
\usepackage{booktabs} % For nicer table rules
\usepackage{array}  % Extended col types if needed
\usepackage{multirow}
\usepackage{multicol} % For captions
\usepackage[most]{tcolorbox} % For colored boxes
\usepackage{enumitem} % For custom list environments
% If the title and author information does not fit in the area allocated, uncomment the following
%
%\setlength\titlebox{<dim>}
%
% and set <dim> to something 5cm or larger.

\setlength{\parindent}{0pt} % No indentation

% Congfigure the table of contents and section numbering
\setcounter{tocdepth}{4}
\setcounter{secnumdepth}{4}

% Configure \Cref to use \S for sections and \S\S for subsections
\usepackage[titletoc]{appendix}
\usepackage{cleveref}
\makeatletter
\newcommand{\crefnames}[3]{%
  \@for\next:=#1\do{%
    \expandafter\crefname\expandafter{\next}{#2}{#3}%
  }%
}
\makeatother
%\crefname{section}{Section}{Sections}
\crefnames{app}{Appendix}{Appendices}
\crefname{appendix}{Appendix}{Appendices}
\Crefname{appendix}{Appendix}{Appendices}
% \crefnames{part,chapter,section}{\S}{\S\S}
% \crefnames{paragraph}{\P}{\P\P}
% \crefname{abstract}{abstract}{abstracts}

\newcommand{\numlist}[1]{(\textbf{#1})}

\newcounter{para}
\newcommand\numpara{\par\refstepcounter{para}{\thepara}.\space\textbf}


% need to add this to avoid "undefined control sequence" error in .bib
% Holmes paper
\newcommand{\recorder}{\textregistered}

% specific to this paper
\newcommand{\links}{\textbf{\textsc{Links}}\space}
\newcommand{\linksys}{\textbf{\textsc{LinkSYS}}\space}
\newcommand{\studentmodel}{Gemma-3-1b-it\space}
\newcommand{\teachermodel}{DeepSeek-R1\space}
\newcommand{\judgemodel}{OpenAI's o3-mini\space}
\newcommand{\mnemonic}{$m$\space}
\newcommand{\vocabulary}{$v$\space}
\newcommand{\eg}{$e$\space}

\title{\textbf{LINKS}: Generate linguistically grounded mnemonic devices for English vocabulary learning with reasoning, multilingual LLMs
}

\author{%
  My (Chiffon) Nguyen \\
  College of Computational Sciences \\
  Minerva University \\
  chiffonng@uni.minerva.edu
}

\begin{document}
\begin{titlepage}
\centering
{\scshape\LARGE Minerva University \par}
\vspace{1cm}
\begin{center}
  \includegraphics[width=0.4\linewidth]{minerva/minerva_logo.pdf}
\end{center}
{\scshape\Large Capstone: Class of 2025 \par}
\vspace{1.5cm}
{\huge\bfseries LINKS: Generate linguistically grounded mnemonic devices for English vocabulary learning with reasoning, multilingual LLMs \par}
\vspace{2cm}
{\scshape\large Tra My (Chiffon) Nguyen \par}

\vfill
submitted in partial fulfillment of the requirements for the degree of \\ Bachelor of Science in Computational Sciences \par
\vspace{2cm}
{\large Capstone Committee \par}
Dr. Patrick Watson \\
Dr. Philip Sterne \\
\vspace{2cm}
{\large \today\par}
\end{titlepage}

\onecolumn
\section*{Executive Summary} \label{sec:exec-summary}

Learning new vocabulary is a universal challenge faced by language learners worldwide. Traditional approaches often rely on rote memorization, which can be tedious and ineffective for many learners. Our research tackles this problem by developing an innovative approach to vocabulary learning through "linguistically grounded mnemonics", memory aids that leverage a word's linguistic features such as etymology, sound patterns, or word structure to create meaningful connections that make words easier to remember.

I've created a specialized AI system called \linksys that generates mnemonics for English vocabulary words. Building on the keyword method that associates new, foreign words with familiar words, our system goes a step further by creating mnemonics that incorporates linguistic rich linguistic characteristics of words, helping learners understand not just what words mean but why they mean what they do. For example, instead of simply memorizing that "adumbrate" means "to outline or foreshadow," our system explains that it comes from Latin roots "ad-" (to, on) + "umbra" (shade), creating a mnemonic that connects the word's meaning to its origin. The system generates mnemonics that are not only linguistically rich but also memorable.

My approach combines advanced AI techniques to distill linguistic reasoning into a compact, accessible model that outperforms larger systems while requiring fewer computational resources. This makes it practical for deployment in educational settings, even with limited technology budgets. Through rigorous testing, we found that learners consistently preferred our system's mnemonics over those generated by standard AI models. This research represents a step toward more effective, personalized vocabulary learning tools that can adapt to different learning styles and preferences, potentially transforming how we approach language education.

Tags: computational linguistics, natural language processing, large language model, language education, english as a foreign language, vocabulary acquisition, synthetic data generation.

\subsection*{AI Statement} \label{sec:ai-statement}

The main idea of this project is to use large language models (LLMs) to generate mnemonic devices for English vocabulary learning. Such AI usage is documented in the main paper.

I extensively used Claude 3.7 Sonnet connected with my codebase to \numlist{1} generate working Python code for major features and plots, \numlist{2} debug my code and \numlist{3} iteratively improve my codebase with best practices including refactoring, modularization, and documentation. I also used Claude to aid producing this paper by \numlist{4} generating LaTex figures and tables, such as \Cref{fig:good-bad-mnemonics} and \Cref{tab:significance-llm-judge}, \numlist{5} explaining new methods used, especially GRPO (\Cref{app:grpo}), and \numlist{6} rewriting some sections of the paper to improve clarity and conciseness, particularly Abstract and \Cref{sec:discussion}.

There were several instances where AI failed to help, mostly due to \numlist{1} the usage of AI-related knowledge and packages that are recently released or updated, such as \verb|trl|'s \verb|GRPOTrainer| class, or and \numlist{2} low-frequency knowledge, such as some TeX formatting or lower-level software or hardware issues. An example: I failed to send API requests to \teachermodel from \verb|curator| but not from \verb|openai| library, due to SSL certificate issue. The root cause was a conflict between different Python versions installed globally on my laptop, one from \verb|homebrew| and one from Python distribution.

\subsection*{Important Notes} \label{sec:important-notes}

Due to limited compute, some experiments conducted are small-scale and need more data for robust validation and conclusion. However, the codebase is reproducible and scalable when there is more compute. All links, including this paper source .tex, is included on \hyperlink{https://github.com/chiffonng/mnemonic-gen}{Github}.

This project went through multiple technical iterations behind the scene, from supervised finetuning (Nov 2024) to group releative policy optimization (Feb 2025) (for details refer to \Cref{app:previous-iterations}). The main paper only discussed the final iteration, which is the most promising one. The other iterations are not included in the paper but their implementation is available in the codebase in forms of Jupyter Notebooks and old pull requests.

\clearpage

\tableofcontents


\maketitle
\begin{abstract}

To acquire advanced vocabulary (CEFR B2+), English learners often use mnemonic devices, memorable associations linking a new concept to learned concepts to improve memory and recall. We consolidated characteristics of good mnemonics and propose the usage and creation of \textbf{linguistically grounded mnemonics}, which better link to the target vocabulary, improving long-term retention and linguistic knowledge. We investigated whether Large Language Models can consistently help write such effective mnemonics, with two different settings: in-context learning, and reasoning distillation. Concretely, we first measured different prompting strategies with frontier models and generated \links, a synthetic dataset of 2000 triplets of \textit{reasoning trace, mnemonic, and example sentence} for 2000 vocabulary useful for TOEFL iBT \footnote{Internet-based Test of English as a Foreign Language}, IELTS Academic \footnote{International English Language Testing System}, and SAT\footnote{Scholastic Aptitude Test}. Second, using a subset of \links, we distilled linguistic reasoning from the \textit{teacher model}, \teachermodel, to the \textit{student model}, \studentmodel\footnote{\url{https://huggingface.co/collections/google/gemma-3-release-67c6c6f89c4f76621268bb6d}} , with online reinforcement learning. The trained, quantized model can be served with local applications such as OpenWebUI (interface) and Ollama (command-line).

Preliminary evaluation shows

The project examplifies that carefully designed NLP systems can generate resources for language learning, for both classroom settings and self-study.\footnote{\url{https://github.com/chiffonng/mnemonic-gen}}

\end{abstract}

\section{Introduction} \label{sec:intro}

%%%% Figure Highlight the difference between keyword "mnemonic" and "linguistically grounded mnemonics" with an example
%%% (e.g., \textbf{preposterous} can be broken down as pre- (before) + poster (after) + ous. Anything that comes both before and after is preposterous)

Vocabulary acquisition challenges many English language learners, particularly at upper intermediate to advanced levels where abstract and academic vocabulary predominates. For effective vocabulary learning, deeper linguistic engagement that connects new vocabulary to existing knowledge is essential. Large Language Models (LLMs) have demonstrated capabilities as knowledge bases and creative text generators, suggesting their potential for automated language learning assistance.

Previous work explored automated mnemonic generation through computational methods using the \textbf{keyword method} \citep{atkinsonApplicationMnemonicKeyword1975}. \citet{savvaTransPhonerAutomatedMnemonic2014} and \citet{OzbalAUTOMATION2014} generated keywords of phonetic and orthographic similarities in the native language for foreign language vocabulary, across multiple languages. \citet{LeeSMARTPHONE2023} extended this work and utilized LLMs to produce phonetically similar keywords and visual cues and \citet{LeeEXPLORING2024} prompted LLMs to generate multiple mnemonic candidates and evaluate them based on imageability and coherence. Most recently, \citet{BalepurSMART2024} fine-tuned LLaMA-2-70B on 800 crowd-sourced mnemonics and aligned outputs with learner preferences and learning outcomes.

These approaches, however, have primarily focused on the keyword method while neglecting other rich linguistic knowledge embedded in LLMs. Additionally, they typically deliver generated mnemonics passively to learners or aggregated learner preferences, which may not align with individual learning styles. For instance, \textsc{Smart} \citep{BalepurSMART2024} was integrated into a Spaced Repetition System (SRS) that tested learners' knowledge, it did not actively engage learners in the mnemonic creation process. \textsc{Smart} also aggregated learner preferences and outcomes, potentially missing alignment with individual learning styles. Research shows that self-created mnemonics lead to more effective and longer-lasting vocabulary retention \citep{madanExploringWordMemorability2021}.

Our research investigates whether LLMs can generate linguistically grounded mnemonics that leverage multiple linguistic features beyond simple keyword associations. Our key questions include: \numlist{1} Can LLMs, particularly reasoning ones, leverage various linguistic features for mnemonic generation without further training? \numlist{2} If so, can we distill these capabilities to smaller, locally deployable models? \numlist{3} Are LLM-generated mnemonics comparably helpful to human-generated mnemonics in terms of memorability and retention?

In this paper, we demonstrate that LLMs can generate linguistically grounded mnemonics through reasoning and creative writing (\Cref{sec:icl-performance}). We present \links, a synthetic dataset of 2000 triplets of \textit{reasoning trace, mnemonic, and example sentence} for vocabulary terms, suitable for integration with SRS or language learning applications. \numlist{3} We distill the reasoning capabilities of a teacher model (\teachermodel) into a smaller student model (\studentmodel) (\Cref{sec:distillation}). The resulting model, \linksys, can be deployed locally, enabling users to generate mnemonics without relying on external APIs.

\section{Background} \label{sec:background}

We assume a background on LLMs, including their transformer-based architecture (\Cref{app:llm-transformer}), in-context learning (\Cref{sec:icl-performance}), and reinforcement learning (full preliminaries are provided in \Cref{app:technicality}). We briefly review the literature on mnemonic devices for vocabulary learning and the use of LLMs in linguistic tasks.

\subsection{Mnemonic devices for vocabulary learning} \label{sec:mnemonic-review}

% TODO: Add a figure for what good mnemonics means. Here is a placeholder
% \begin{figure*}[htb]
%   \includegraphics[width=\linewidth]{figures/good_bad_mnemonics.pdf}
%   \caption{Non-exhausitive list of characteristics of a good mnemonic, inferred from \citetext{\citealp{BalepurSMART2024}; \citealp{CamposUSING2011}; \citealp{luExplorationMnemonicsESL2015}; \citealp{SariogluUSE2024}}.}
%   \label{fig:good-bad-mnemonics}
% \end{figure*}

% Define custom colors


% Define custom box styles
\tcbset{
    goodbox/.style={
        colback=goodlight,
        colframe=goodgreen,
        fonttitle=\bfseries\color{white},
        coltitle=white,
        colbacktitle=goodgreen,
        enhanced,
        attach boxed title to top center={yshift=-1mm},
        boxed title style={sharp corners},
        top=2mm,
    },
    badbox/.style={
        colback=badlight,
        colframe=badred,
        fonttitle=\bfseries\color{white},
        coltitle=white,
        colbacktitle=badred,
        enhanced,
        attach boxed title to top center={yshift=-1mm},
        boxed title style={sharp corners},
        top=2mm,
    }
}

% Start the figure environment
\begin{figure*}[htb]
\centering
\footnotesize
% First row - Good mnemonic + characteristics
\begin{minipage}{0.66\textwidth}
    \begin{tcolorbox}[goodbox, title=Good Mnemonic]
        \textcolor{red}{\textbf{preposterous}}: \textcolor{goodgreen}{comes from pre- ("before") + post ("after") + -ous}, meaning reversed or absurd. \textcolor{orange}{An event cannot happen both pre- (before) and post- (after) to us, it is prespoterous!}
    \end{tcolorbox}
\end{minipage}
\hspace{0.5em}
\begin{minipage}{0.3\textwidth}
    \begin{itemize}[leftmargin=*, nosep]
        \item \vocab is used correctly in \mnem
        \item Clear \assoc linking \vocab and \mnem
        \item Strong \assoc
        \item \mnem uses similar or lower vocabulary than \vocab
        \item \mnem is memorable
    \end{itemize}
\end{minipage}

\vspace{0.3cm}

% Second row - 3 bad mnemonics
\begin{minipage}{0.33\textwidth}
    \begin{tcolorbox}[badbox, title=Incorrect Definition]
        Preposterous means very important or significant.
    \end{tcolorbox}
\end{minipage}%
\begin{minipage}{0.33\textwidth}
    \begin{tcolorbox}[badbox, title=Circular Association]
        Preposterous sounds like preposterous.
    \end{tcolorbox}
\end{minipage}%
\begin{minipage}{0.33\textwidth}
    \begin{tcolorbox}[badbox, title=Weak Association]
        Preposterous sounds like prosperous. Preposterous people usually make prosperous business decisions.
    \end{tcolorbox}
\end{minipage}

\vspace{0.3cm}

% Third row - 3 more bad mnemonics
\begin{minipage}{0.33\textwidth}
    \begin{tcolorbox}[badbox, title=Difficult Vocabulary]
        Preposterous means ludicrously implausible or contrary to conventional hierarchies of logical induction.
    \end{tcolorbox}
\end{minipage}%
\begin{minipage}{0.33\textwidth}
    \begin{tcolorbox}[badbox, title=Too Abstract]
        Preposterous describes logical fallacies where the premise negates itself through temporal displacement.
    \end{tcolorbox}
\end{minipage}%
\begin{minipage}{0.33\textwidth}
    \begin{tcolorbox}[badbox, title=Offensive Content]
        Preposterous contains "post" which reminds me of [inappropriate culturally-specific reference].
    \end{tcolorbox}
\end{minipage}

\caption{Characteristics of good mnemonics, and examples of bad mnemonics. We propose VAM/VEM model, where a good mnemonic must have three components: \vocabulary (\vocab), \association (\assoc) (or explanation ($e$)), and \mnemonic (\mnem), with characteristics listed above. These characteristics are also available in list (\Cref{app:mnemonic-characteristics})}
\label{fig:good-bad-mnemonics}
\end{figure*}


Mnemonic devices are systematic mental techniques that enhance memory through meaningful associations between new information and pre-existing knowledge \citep{PintrichROLE2002, pressleyMnemonicKeywordMethod1982}. For vocabulary acquisition, the keyword method has been widely studied, involving the creation of acoustically or orthographically similar keywords to the target vocabulary, followed by an association between these keywords and the word's meaning \citep{atkinsonApplicationMnemonicKeyword1975, wangKeywordMnemonicRetention1992}.

While the keyword method has shown effectiveness for concrete vocabulary \citep{avilaExploringNewApplications1996, camposImportanceKeywordGenerationMethod2004a}, research indicates it becomes less effective for abstract terms \citep{fothMnemonicTechniqueEffectiveness1973, CamposLIMITATIONS2003} and may even result in longer retrieval times compared to rote learning \citep{vanhellKeywordMnemonicsRote1997}. Additionally, experienced language learners often find traditional rote learning more effective than keyword mnemonics, suggesting that mnemonic effectiveness varies with learner proficiency \citep{vanhellKeywordMnemonicsRote1997, CamposUSING2011}.

More linguistically sophisticated approaches, such as etymology-based mnemonics, could provide deeper encoding and potentially stronger retention for abstract vocabulary \citep{piersonUsingEtymologyClassroom1989, akarslanEffectsTeachingWord2019, gangavarapuUsingEtymologyVocabulary2024}. These approaches leverage morphological, etymological, and semantic properties of words, creating more meaningful associations that align with how language is naturally structured \citep{zhangApplicationEtymologySemantic2013}.

According to psycholinguistics, characteristics make mnemonics more memorable include references to animate entities, potential usefulness, and connecion to concrete visual imagery \citep{madanExploringWordMemorability2021, ledingAdaptiveMemoryAnimacy2019}. Additionally, emotionally charged associations create stronger memory traces than neutral ones \citep{altarribaConcretenessContextAvailability1999}. The effectiveness of mnemonics also depends on the depth of processing involved, with deeper linguistic analysis creating more robust memory traces \citep{rankinAgePresentationRate1983, SariogluUSE2024}.

We explore these principles in our work, focusing on how LLMs can generate mnemonics that incorporate good characteristics (\Cref{fig:good-bad-mnemonics}) and leverage linguistic features (\Cref{tab:linguistic-features}).

\subsection{LLMs: linguistic competence, reasoning, and creativity} \label{sec:llm-linguistic-competence}

Significant advancements have been made in enhancing the reasoning capabilities of large language models (LLMs), particularly in mathematical and scientific domains where problems have unique correct answers. Several prompting techniques were introduced to make LLMs learn from demonstrations (or "shots") or produce explicit step-by-step thinking processes to improve their reasoning, notably few-shot prompting \citep{brownFewShotLearners2020}, chain-of-thought (CoT) \citep{weiChainofThoughtPromptingElicits2022}, self-consistency \citep{wangSelfConsistencyImprovesChain2022}, zero-shot reasoning \citep{kojimaZeroShotReasoners2022}, analogical reasoning (automated few-shot CoT) \citep{YasunagaLLMAnalogicalReasoners2023}. Post-training techniques such as reinforcement learning from human feedback (RLHF) \citep{ouyangRLHF2022} and CoT data \citep{DeepSeek-AIDEEPSEEKR12025} further endows LLMs with instruction-following and reasoning capabilities. However, LLMs still struggle with complex reasoning tasks that require multiple steps, abstract thinking \citep{weiChainofThoughtPromptingElicits2022} or low-frequency knowledge \citep{kandpalLongTailKnowledge2023,sunHeadtoTailHowKnowledgeable2024}.

LLMs' linguistic competence and reasoning are less studied. Recent studies have explored LLMs' linguistic competence, defined as their ability to understand and apply language rules and patterns \citep{waldisHOLMES2024}. LLMs typically perform better on formal linguistic competence tasks, such as morphology and syntax, than on functional linguistic competence tasks, such as semantics, discourse \citep{KhoujaLINGOLYTOO2025} or phonology \citep{suvarnaPhonologyBenchEvaluatingPhonological2024}. Their competence is influenced by model architecture, with encoder-based models often outperforming decoder-only models, and larger models generally showing better linguistic understanding \citep{waldisHOLMES2024}. Instruction tuning has been shown to improve performance on linguistic tasks, though sometimes at the expense of deeper language understanding \citep{waldisHOLMES2024,yinDidYouRead2023}.

LLMs can also perform inductive multilingual reasoning, primarily demonstrated through inferring rules in linguistic puzzles as seen in International Olympiad in Linguistics, especially when provided with analogical demonstrations \citep{RamjiINDUCTIVE2024}. However, its reasoning remains inconsistent, with performance varying across minor problem perturbations, suggesting that it may not fully understand the underlying linguistic principles and memorize it \citep{RamjiINDUCTIVE2024,KhoujaLINGOLYTOO2025}. This inconsistency is also observed in other reasoning tasks, such as mathematical reasoning, where LLMs can produce correct answers but often fail to provide coherent explanations \citep{weiChainofThoughtPromptingElicits2022}.

For mnemonic generation specifically, previous work has explored using LLMs for automated keyword mnemonic generation \citep{LeeSMARTPHONE2023, LeeEXPLORING2024, BalepurSMART2024}, but these approaches have primarily focused on phonetic similarity rather than leveraging the broader linguistic knowledge embedded in LLMs. Our work extends these efforts by exploring how LLMs can use linguistic reasoning abilities to incorporate multiple linguistic features to generate mnemonic devices.


\section{In-context learning performance} \label{sec:icl-performance}

We first investigated how effectively frontier LLMs could generate linguistically grounded mnemonics through in-context learning. This exploration aimed to establish baseline performance and identify optimal prompting strategies before progressing to more resource-intensive approaches.

\subsection{Experimental setup}
We compared different in-context learning approaches to understand how they affect mnemonic generation quality. Using a test set of 50 vocabulary words from SAT and TOEFL exams, we evaluated four distinct prompting strategies with \xteachermodel (multimodal) and \teachermodel (reasoning) \citep{DeepSeek-AIDEEPSEEKR12025,DeepSeekV32025}.

Given a vocabulary \vocab and a list of mnemonic characteristics (\Cref{fig:good-bad-mnemonics}), we designed a prompt $p$ and prompted the model $M$ to generate a \lgm \mnem for \vocab. To elicit reasoning in \xteachermodel, we added the phrase "Let's think step by step" to all the prompts, a common strategy in prompting LLMs for reasoning tasks \citep{weiChainofThoughtPromptingElicits2022}. We repeated this process for 50 vocabulary \vocab, 4 prompts $p$, and 2 models $M$, resulting in 400 total API requests. The prompts were designed to elicit different levels of linguistic reasoning and mnemonic generation strategies, with full description in \Cref{app:prompt-usage}. We used \verb|curator| \citep{BespokeLabBESPOKE2025} with \verb|litellm| orchestration layer to standardize API calls, manage rate limits, and handle retries across experiments.

%% The expected workflow for the models are combining the role of linguist and English language educator:
% input: digest task instructions -> digest examples
% reasoning: analyze vocabulary and its linguistic features -> assess which features are the most relevant to learners
% output: construct a mnemonic based on the selected features and meet other characteristics of good mnemonics \Cref{fig:good-bad-mnemonics}.

We evaluated the outputs based on two criteria: \numlist{1} lingusitic grounding, whether there is at least one of the linguistic features (\Cref{tab:linguistic-features}) and \numlist{2} overall quality, which we manually assessed using the given rubric (\Cref{fig:good-bad-mnemonics}).

\subsection{Results} \label{sec:icl-results}

\begin{figure}[htb]
  \centering
  \includegraphics[width=\linewidth]{figures/prompt_comparison.pdf}
  \caption{Comparison of prompting methods (details in \Cref{app:prompt-usage}). Y-axis shows percentage of \lgms generated out of 50 requests sent for 50 vocabulary \vocab for each prompt $p$.}
  \label{fig:prompting-methods}
\end{figure}

As shown in \Cref{fig:prompting-methods}, \teachermodel consistently outperformed \xteachermodel across all prompting strategies, confirming our hypothesis that models specialized for reasoning tasks are better suited for linguistic analysis.

The vanilla prompt, which simply requested a \lgm for a vocabulary \vocab, produced \lgms in only 20\% of cases for \xteachermodel and 34\% for \teachermodel.

Changing the terminology from "mnemonic" to "memory cue" in increased the proportion of linguistically grounded responses to 26\% and 40\% respectively. We have a weak hypothesis that the word "mnemonic" may carry pre-training biases that associate it primarily with acronyms or simple keyword methods \citep{hackmannWordImportanceExplains2024}.

We also noted \teachermodel tended to generate lengthy, divergent reasoning traces, so we instructed it to stop analysis of linguistic features when it found a good enough mnemonic. When combining this instruction with structured output format, \xteachermodel produced \lgms in 35\% of cases, while \teachermodel achieved 52\%. This confirms that the model's performance didn't drop with shortened reasoning \citep{xuChainDraftThinking2025}, but further improved with explicit task instructions \citep{yinDidYouRead2023}.

The structured 10-shot CoT prompt, which included examples of linguistic reasoning before mnemonic generation, achieved the highest success rates of 68\% for \xteachermodel and 84\% for \teachermodel. This approach effectively guided the models to balance lingusitic grounding and memorability of mnemonics from the examples, preventing them from generating overly complex or abstract mnemonics.

Our qualitative analysis revealed several patterns in the generated mnemonics: \numlist{1} LLMs defaulted to surface-level associations rather than deeper linguistic analysis. \numlist{2} Both models favored etymology and morphology in their linguistic analysis, but they usually combined with phonetic or orthographic keywords in generated mnemonics \numlist{3} Several mnemonics are linguistically grounded but not memorable, despite the presence of both requirements in the prompt.

These findings indicated that while LLMs possess substantial linguistic knowledge for English language accessible through appropriate prompting, they benefit from structured guidance to apply this knowledge effectively for mnemonic generation. The strongest performance from CoT prompting suggested that reasoning elicitation is crucial for high-quality, linguistically grounded mnemonics.

Based on these insights, we selected the 10-shot CoT prompting approach to generate our synthetic dataset for model distillation, as described in the next section.


\section{Knowledge and reasoning distillation} \label{sec:distillation}

\begin{figure}[htb]
  \centering
  \includegraphics[width=\linewidth]{figures/pipeline.pdf}
  \caption{\linksys pipeline. The pipeline consists of two main components: (1) CoT data generation and (2) model distillation. We generate a dataset of mnemonics with reasoning traces using a large language model (LLM) as a teacher. Then we distill the reasoning capabilities of the teacher model into a smaller student model using GRPO \Cref{app:grpo}. Finally, we obtain \linksys, a smaller model that can generate mnemonics through linguistic reasoning.}
  \label{fig:distillation}
\end{figure}

\subsection{Preparation} \label{sec:data-prep}
We first created a comprehensive training dataset. Following best practices in synthetic data generation with LLMs \citetext{\citealp{longLLMsDrivenSyntheticData2024b}, \citealp{openthoughtsteamOpenThoughts2025}}, we designed a data construction pipeline with key components (\Cref{fig:distillation}).

\paragraph*{Vocabulary selection} \label{sec:vocab-selection}
We collected 5,000 distinct words from three main sources: English as a foreign language tests (TOEFL iBT, IELTS Academic), standardized tests with verbal reasoning (SAT, GRE), and CEFR C1/C2 word lists. This ensured coverage of academic and abstract vocabulary that would benefit from mnemonic devices. After fuzzy-matching deduplication (threshold 95\%), we refined the dataset and performed stratified random sampling (stratified by linguistic features used) to obtain 2,000 distinct vocabulary for post-training.

\paragraph*{Prompt design and model selection} Based on \cref{sec:icl-performance}'s findings, we designed system and user prompts, focusing on the 10-shot CoT approach that demonstrated superior performance in eliciting linguistic reasoning and creating memorable, \lgms.

\subsection{Dataset generation} \label{sec:data-gen}

Using our optimized prompts and vocabulary list, we generated the \links dataset of approximately 2,000 entries, each containing:

\begin{equation}
(\tau_i, m_i, e_i) \in \mathcal{D}
\end{equation}

where $\tau_i$ represents the reasoning trace exploring linguistic features, $m_i$ the generated mnemonic, and $e_i$ an example sentence for vocabulary $i$. The dataset creation process can be formalized as:

\begin{equation}
(\tau_i, m_i, e_i) = f_{\text{teacher}}(v_i; p_{\text{CoT}})
\end{equation}

where $f_{\text{teacher}}$ is the teacher model function, $v_i$ is vocabulary $i$, and $p_{\text{CoT}}$ is our 10-shot CoT prompt.

\subsection{Quality control} To ensure the quality of the generated mnemonics, we implemented a multi-step validation process. We first filtered out any entries that did not meet our structured output format or contained incomplete reasoning traces. We then performed a manual review of a random sample of 200 entries to assess the linguistic grounding and coherence of the mnemonics. This review process involved checking for clear connections between the vocabulary and the mnemonic, as well as ensuring that the example sentence accurately reflected the vocabulary's meaning.

\subsection{Training and inference} \label{sec:training-inference}
To transfer the linguistic reasoning capability to a smaller model, we implemented a distillation process using Group Relative Policy Optimization (GRPO).

We selected Google's \studentmodel \citep{GemmaTeamGEMMA2025} as our student model due to its balance of performance and size (1 billion parameters). It can handle general-purpose tasks in multiple languages, and demonstrate instruction-following abilities. We used a 4-bit quantized version to further reduce memory requirements while maintaining performance \citep{dettmersQLoRAEfficientFinetuning2023}.

\paragraph*{Group Relative Policy Optimization (GRPO)} We employed GRPO \citep{DeepSeek-AIDEEPSEEKR12025} to distill the reasoning capabilities into the student model. For each input prompt $q$ generating a vocabulary $v$, the model produces $G$ candidate outputs:

\begin{equation}
O_q = \{o_1, o_2, \ldots, o_G\}
\end{equation}

These outputs are evaluated using reward functions $r_j$ that output scores, with the overall reward for output $i$ calculated as:

\begin{equation}
r_i = \sum_{j=1}^{J} w_j r_j(o_i, v)
\end{equation}

where $w_j$ is the weight for reward function $j$. We provide more technical details in \Cref{app:grpo}, including the policy loss to be minimized.

We defined three reward functions $r$ that encode essential characteristics of effective mnemonics:
\numlist{1} adherence to the structured format with reasoning, mnemonic, and example,
\numlist{2} usage of the target vocabulary in the mnemonic, penalizing bad mnemonics such as acronyms, and
\numlist{3} explicit incorporation of linguistic features in \Cref{tab:linguistic-features} or a reasonable custom feature.

We assigned higher weights to criterion 3, and generated $G=2$ candidates per training example to enable learning from comparisons. Training was performed on a single NVIDIA H100 GPU for approximately 4 hours (more details in \Cref{app:grpo-config}).

\paragraph*{Low-Rank Adaptation (LoRA)} We trained \studentmodel using GRPO (\Cref{app:peft}) wrapped in LoRA layers (\Cref{app:lora-config}) to reduce the number of trainable parameters and rank-stabilized LoRA that maintains stability for adapters with higher ranks.

\section{Evaluation} \label{sec:evaluation}
We evaluated the performance of \linksys in two main ways: \numlist{1} qualitative grading with LLM-as-a-judge and \numlist{2} pairwise preference using double-blind annotations. The first method involved using another LLM, \judgemodel, to evaluate the quality of mnemonics generated by our model, while the second involved annotators comparing mnemonics generated by the base model, \studentmodel, and our model, \linksys.

\begin{table*}[!htb]
\centering
\caption{Metric comaparison between the base (\studentmodel) and \linksys models}
\label{tab:significance-llm-judge}
\begin{tabular}{lccccc}
\toprule
\textbf{Evaluation Metric} & \textbf{Base} ($\mu$) & \linksys ($\mu$) & $\mu$ diff. & \textbf{p-value} & \textbf{Effect size} \\
\midrule
Correct vocabulary usage (bool) & 0.755 & 0.795 & +0.040 & 0.077 & -- \\
Linguistic grounding (bool) & 0.670 & 0.735 & +0.065 & \textbf{0.009} & -- \\
Association strength (1-5) & 2.845 & 3.070 & +0.225 & \textbf{<0.001} & 0.464 \\
Clarity (1-5) & 3.155 & 3.410 & +0.255 & \textbf{<0.001} & 0.481 \\
Memorability (1-5) & 2.600 & 2.900 & +0.300 & \textbf{<0.001} & 0.566 \\
\bottomrule
\end{tabular}
\begin{minipage}{\textwidth}
\vspace{0.1em}
\small
\textit{Note:} Bold p-values indicate statistically significant differences ($p < 0.05$). Effect sizes are Cohen's d, where 0.2 is small, 0.5 is medium, and 0.8 is large. Boolean metrics (first two rows) do not have effect sizes.
\end{minipage}
\end{table*}


\subsection{Qualitative grading with LLM judge} \label{sec:qualitative-llm-judge}

We designed a structured evaluation protocol using \judgemodel as a judge to assess mnemonic quality. The LLM judge evaluated 200 pairs of mnemonics from our test set, with each pair generated by the base model (\studentmodel) and our model \linksys.

We asked the judge to score each mnemonic independently on four metrics from our VAM model (\Cref{fig:good-bad-mnemonics}):
\numlist{1} whether the vocabulary is used correctly in the mnemonic,
\numlist{2} strength of association between the vocabulary and the mnemonic,
\numlist{3} how clear and easy to understand the mnemonic is,
\numlist{4} how memorable the mnemonic is, considering factors like concreteness, imageability, and distinctiveness. The last metric is \numlist{5} whether the mnemonic is linguistically grounded, meaning it incorporates linguistic features such as phonetics, morphology, or etymology.

Each criterion was evaluated on a binary scale (correct usage, linguistic grounding) or a 5-point Likert scale (association, clarity, memorability). The judge was instructed to provide six-field output: the score for each metric, and a brief reasoning for those scores.

We also calculated whether the difference in ratings was statistically significant using \numlist{1} a Wilconoxon signed-rank test for Likert ratings with paired samples and \numlist{2} a McNemar's test for paired boolean ratings. We set the significance level at 0.05.

\begin{figure}[htb]
  \centering
  \includegraphics[width=\linewidth]{figures/boolean_comparison.pdf}
  \caption{LLM-as-a-judge evaluation for boolean metrics (true/false): correct usage of vocabulary in mnemonic and linguistic grounding of mnemonic. \linksys shows improvement in both metrics, with notable gains in linguistic grounding (68\% vs. 82\%).}
  \label{fig:llm-judge-boolean}
\end{figure}

\begin{figure}[htb]
  \centering
  \includegraphics[width=\linewidth]{figures/likert_distribution.pdf}
  \caption{LLM-as-a-judge evaluation for 5-point Likert scale metrics. \linksys shows improvements across all three metrics, with the most significant gains in memorability (mean of 2.6 vs. 2.9)}
  \label{fig:llm-judge-likert}
\end{figure}

We reported the distribution of \judgemodel's ratings in \Cref{fig:llm-judge-boolean,fig:llm-judge-likert}, and summarized the difference in performance between the base model (\studentmodel) and our model in \Cref{tab:significance-llm-judge}. The results indicate that \linksys outperforms the base model across all evaluation metrics. Four of the five metrics showed statistically significant improvements ($p < 0.05$). The most substantial improvement was observed in memorability, with a mean difference of $+0.3$ points and a medium effect size (Cohen's $d = 0.566$). Clarity and semantic association also showed significant improvements with medium effect sizes.

While correct vocabulary usage showed a positive trend, this difference was not statistically significant ($p = 0.077$). This suggests that both models were already competent at using vocabulary correctly, with less room for improvement in this area.

\subsection{Pairwise preference using blind annotations} \label{sec:pairwise-preference}

To directly compare the quality of mnemonics, we conducted a blind annotation study. We randomly selected 100 mnemonics from our test set, with 50 each generated by the base model (\studentmodel) and our model (\linksys). Two annotators (the author and an LLM) were asked to evaluate the mnemonics in pairs, choosing the better mnemonic for each pair while being blinded to the model used and the order of mnemonic presented in a pair\footnote{We acknowledge both using the author and using an LLM for evaluation as limitations in \Cref{sec:limitations}.}. This approach allowed us to understand the relative performance of each model. The annotators were instructed to consider the same criteria as in the LLM-as-a-judge evaluation, and indicated which mnemonic they preferred and why. More details in \Cref{app:annotation}.

\Cref{fig:pairwise-preference} reveals a strong preference for mnemonics generated by \linksys over those from the base model (64\% vs. 36\%). Annotators particularly valued mnemonics that incorporated multiple linguistic features and provided clear, concrete associations to abstract concepts.

\begin{figure}[htb]
  \centering
  \includegraphics[width=\linewidth]{figures/model_comparison.pdf}
  \caption{Pairwise preference using double-blind annotation. Y-axis shows the percentage of preference for each mnemonic generation method.}
  \label{fig:pairwise-preference}
\end{figure}

\section{Discussion} \label{sec:discussion}


\section{Conclusion} \label{sec:conclusion}
This paper introduced

\subsection{Limitations} \label{sec:limitations}
Despite promising results, several limitations warrant acknowledgment. (1) Resource constraints limited the scale of our experiments (\Cref{sec:icl-performance,sec:evaluation}) and the limited number of annotators in our double-blind study (\Cref{sec:pairwise-preference}) may have introduced bias in the evaluation process. Future work should consider using a larger and more diverse set of judges and annotators to ensure robustness and generalizability of the findings.

(2) Our evaluation focused primarily on intermediate measures of mnemonic quality rather than direct assessment of learning outcomes. Future work should include longitudinal studies measuring actual vocabulary retention using LLM-generated mnemonics.

(3) The use of LLM-as-a-judge for qualitative grading and pairwise preference evaluation may introduce biases. We acknowledged the potential for biases in LLMs, including self-enhancement bias \citep{panicksseryLLMEvaluatorsRecognize2024}, positional bias \citep{wangNotFairEvaluators2024,zhengJudgingLLMasajudgeMTbench2023}, verbosity bias \citep{zhengJudgingLLMasajudgeMTbench2023}, and others. Respectively, we attempted to mitigate these biases by employing a structured evaluation protocol, shuffling the order of mnemonics in a pair, enforcing a structured output format, and controlling for generated mnemonics length by both models in comparison \citep{guSurveyLLMasaJudge2025}. Additionally, we employed a double-blind annotation study with human annotators to validate the results. However, the limited number of annotators (two) may have introduced bias in the evaluation process. Future work should consider using a larger and more diverse set of judges and annotators to ensure robustness and generalizability of the findings.

(4) The focus on English words and English mnemonics may limit the generalizability of our findings to other types of vocabulary (e.g., phrasal verbs, idioms), other languages, and cross-lingual vocabulary-mnemonics. Future research should explore the applicability of our approach to different languages and cultural contexts, such as generating Vietnamese mnemonics to help learn Chinese vocabulary through radicals and cognates.

\subsection{Future Work} \label{sec:future-work}
Future research directions include expanding linguistic annotations to cover a broader range of features and vocabulary types, developing automated methods for linguistic feature extraction, and exploring personalized mnemonic generation that adapts to individual learning preferences and styles. Additionally, integrating multimodal elements that combine visual and textual mnemonics could further enhance learning effectiveness, particularly for concrete vocabulary.
 % includes limitations, conclusion, future work

\section*{Acknowledgements}
I would like to express my gratitude to my capstone committee, Dr. Patrick Watson and Dr. Philip Sterne, for their guidance and support throughout this project. I also want to thank my classmates and friends for their encouragement and feedback during the development of this project.

\section*{Ethics Statement}
This project was conducted in accordance with the ethical guidelines of Minerva University. The dataset used for training and evaluation was generated using an LLM, and a \textbf{subset} of generated mnemonics were reviewed for quality and appropriateness. We acknowledge the potential biases present in the training data and the need for continuous monitoring and improvement of the model's outputs. The final model is designed to be deployed locally, ensuring user privacy and data security.

The generated mnemonics are intended to be used as a supplementary tool for language learners and should \textbf{not} replace other language learning methods. We recommend that users critically evaluate the generated mnemonics and adapt them to their individual learning styles and preferences. We welcome feedback on any aspect of the paper to help improve the model's performance and address any ethical concerns that may arise.

% References
\bibliography{custom}


\begin{appendices}
  \section{Mnemonics: Linguistic Features and Characteristics} \label{app:mnemonics}


\subsection{Full mnemonic characteristiics} \label{app:mnemonic-characteristics}

\begin{itemize}
  \item Clear explanation linking the vocabulary to the mnemonic.
  \item Correct usage and definition of the vocabulary within the mnemonic.
  \item Strong association between the vocabulary and the mnemonic.
  \item Mnemonic is easy to understand, using similar or simpler vocabulary than the target term.
  \item Mnemonic is memorable, incorporating animate or concrete imagery, relevant contexts, or elements that evoke emotional responses.
\end{itemize}

\textbf{One of the following could make bad mnemonics}
\begin{itemize}
  \item Lack one of the three components.
  \item Incorrect definition or usage of the vocabulary.
  \item Circular association where the mnemonic simply repeats the vocabulary without adding meaning. Polysemous words (words with multiple meanings) are not considered circular.
  \item Weak or unclear association between mnemonic and vocabulary.
  \item Use semantically complex or obscure words that are more difficult than the target vocabulary.
  \item Mnemonic is abstract, making it hard to visualize or relate to.
  \item Use offensive or inappropriate language.
\end{itemize}

Unless specifically requested by users, LLMs are prompted to avoid culturally-specific mnemonics, such as those based on idioms or proverbs, as they may not be universally understood. This is particularly important for learners from diverse backgrounds and cultures.
\subsection{Linguistic features} \label{sec:linguistic-features}

\begin{table*}[!htb]
\centering
\caption{Examples of feature categories for English words.}
\label{tab:linguistic-features}
\begin{tabularx}{\textwidth}{l >{\raggedright\arraybackslash}X >{\raggedright\arraybackslash}X}
\toprule
\textbf{feature} & \textbf{description} & \textbf{example} \\
\midrule
\textbf{phonetics} & sound patterns & \emph{apparent} sounds like "a bare Asian." \\
\addlinespace
\textbf{orthography} & written/spelling patterns & \emph{abet} looks like "a + bet." \\
\addlinespace
\textbf{morphology} & modern English forms, including free and bound morphemes & \emph{aggrandize} = a + grand + –ize, to mean to make grander. \\
\addlinespace
\textbf{etymology} & origin and history & \emph{adumbrate} comes from Latin ad- (to, on) + umbra (shade) + ate, to mean foreshadow or outline. \\
\addlinespace
\textbf{semantics} & meaning and semantic relationships & \emph{confound} has similar meaning and history with 'confuse'. \\
\bottomrule
\end{tabularx}
\end{table*}

  
\section{Prompt usage} \label{app:prompt-usage}

All of the following prompts are system prompts for \teachermodel. We also added \Cref{app:mnemonic-characteristics} as mnemonic requirements for all prompts, and added the phrase "Let's think step by step" to the prompt variants sent to \xteachermodel. To conserve tokens, we remove unnecessary words that did not contribute to task instructions, such as modifiers.

\subsection*{Vanilla vs. Alternative Phrasing}

Vanilla prompt
\begin{quotation}
  A mnemonic to help learn and remember meaning of English vocabulary: \{term\}. Mnemonic should have following characteristics:

  [INSERT MNEMONIC REQUIREMENTS HERE].
\end{quotation}

Vanilla-Alternative prompt, with "linguistically grounded" added:
\begin{quotation}
  A \lgm
\end{quotation}

We observed that the vanilla prompts, with the term "mnemonic", often elicit backronyms (i.e. an existing word's letters are expanded into a phrase), initialisms or list. We hypothesized that this was because commonly encountered mnemonics are used for long information and non-linguistic contexts, and the training data reflects that popular use of mnemonic devices. This effect can be mitigated by adding the term "lingusitically grounded" to the prompt, to steer the model towards analyzing linguistic features, especially etymology and morphology.

\subsection*{Structured Output and Task Description}
We found improved performance with prompts that explicitly request broader linguistic analysis and provides structured output format with output descriptions \citep{MishraREFRAMING2022,yinDidYouRead2023}. We also tried to reduce the overthinking tendency in LLMs

\begin{quotation}
Generate a \lgm to help me learn and remember the meaning of English vocabulary: \{term\}.

Analyze linguistic features for this word (etymology, morphology, phonetics, orthography, semantics, etc). Stop linguistic analysis when you have a good linguistic connection. You must use that linguistic feature to form a mnemonic for the word.

Mnemonic should have following characteristics:
[INSERT MNEMONIC REQUIREMENTS HERE].

Provide output in this format:

- linguistic\_feature: chosen linguistic feature for mnemonic

- mnemonic: association + mnemonic

- example: example sentence of the vocabulary in context
\end{quotation}

This approach yielded a higher percentage of mnemonics with clear linguistic association, and better mnemonics overall. We also found that the model was more likely to use the same linguistic feature in the mnemonic as in the analysis, which is a key requirement for our task.

\subsection*{Added CoT examples}

For this prompt, we reused the task instructions and added 10 human-written CoT examples, each demonstrating the process of finding linguistic association of the vocabulary before constructing a mnemonic.

  
\section{Annotation details} \label{app:annotation}

For our double-blind annotation study, we followed best practices from \citet{tsengBestPracticesManaging2020} to ensure unbiased evaluation. The annotation process involved the following steps:

\numlist{1} We randomly selected 50 vocabulary terms from our test set, ensuring representation across different linguistic categories and vocabulary difficulty levels.

\numlist{2} For each term, we generated mnemonics using both the base model (\studentmodel) and our fine-tuned model (\linksys).

\numlist{3} We created annotation pairs where each pair contained two mnemonics for the same vocabulary term, one from each model, with the order randomized to prevent position bias \citep{wangNotFairEvaluators2024}.

\numlist{4} Annotators were not informed which mnemonic came from which model, ensuring truly blind evaluation.

\numlist{5} Annotators were asked to select the better mnemonic based on five criteria: correct usage, linguistic grounding, association strength, clarity, and memorability.

\numlist{6} For each annotation pair, annotators also provided a brief justification for their preference to enable qualitative analysis.

The two annotators were the author and \judgemodel. While this is a limitation of our study that we acknowledge in \Cref{sec:limitations}, we took several measures to reduce potential bias:

\numlist{1} The author completed annotations before running any quantitative analysis to prevent knowledge of statistical patterns from influencing judgment.

\numlist{2} \judgemodel was prompted to evaluate mnemonics according to the criteria in \Cref{fig:good-bad-mnemonics} without any information about which model generated which mnemonic.

\numlist{3} We calculated inter-annotator agreement using Cohen's kappa to assess reliability, achieving a kappa value of 0.72, indicating substantial agreement.

For future work, we plan to expand the annotation team to include language education experts and language learners to provide more diverse perspectives on mnemonic quality.

  
\appsection{Technical preliminaries} \label{app:technicality}

\appsubsection{In-context learning} \label{sec:icl-info}

\appsubsubsection{Chain-of-Thought (CoT) prompting} CoT \citep{weiChainofThoughtPromptingElicits2022} is a prompting technique that encourages LLMs to generate intermediate reasoning steps before arriving at a final answer. This approach has been shown to improve performance on complex tasks by guiding the model through a structured thought process.

\appsubsection{Neural Language Models and Transformer Architecture} \label{app:llm-transformer}

Neural language models are probabilistic frameworks that assign probabilities to sequences of words or subword units, known as tokens. A token is the smallest unit of text that the model processes, which can be as granular as individual characters, subwords, or entire words, depending on the tokenization strategy employed.

Given a sequence of tokens \( \mathbf{x} = (x_1, x_2, \ldots, x_T) \), a language model estimates the joint probability \( P(\mathbf{x}) \) by factorizing it into conditional probabilities:

\begin{equation}
P(\mathbf{x}) = \prod_{t=1}^T P(x_t \mid x_1, x_2, \ldots, x_{t-1})
\end{equation}

At each time step \( t \), the model predicts the next token \( x_t \) based on preceding sequence \( (x_1, x_2, \ldots, x_{t-1}) \). This autoregressive approach enables the generation of coherent text by sequentially predicting subsequent tokens.

The Transformer architecture underpins many state-of-the-art language models due to its efficiency and capability to model long-range dependencies. It utilizes self-attention mechanisms to weigh the relevance of each token in a sequence relative to others, regardless of their positions. The architecture comprises stacked layers, each including multi-head self-attention and position-wise feed-forward networks, facilitating parallelization and effective learning of complex patterns in data.

\subsubsection{Tokenizer} \label{app:tokenizer}

A tokenizer is a preprocessing tool that converts raw text into tokens, aligning the text with the LM's vocabulary. Tokenizers can employ various strategies, such as word-based, character-based, or subword-based tokenization, each with distinct advantages and use cases.

Byte Pair Encoding (BPE) is a subword tokenization algorithm that operates on the byte representation of text, enabling consistent handling of various scripts and special characters. It iteratively merges the most frequent pairs of adjacent bytes to form subword units, constructing a vocabulary that efficiently represents the training corpus. This method allows the tokenizer to decompose rare words into meaningful subword components, enhancing the model's capacity to process diverse and unseen terms.

For instance, the word "preposterous" might be tokenized into subwords like "pre", "poster", and "ous," facilitating the model's understanding and generation of these subwords in novel contexts. This subword granularity enables the model to generalize across morphologically complex words and out-of-vocabulary words, enhancing its robustness and vocabulary coverage. However, not all subwords are valid morphemes, which can limit the model's ability to capture morphological structure accurately. For instance, \texttt{tiktoken} (OpenAI's tokenizer)\footnote{\href{https://platform.openai.com/tokenizer}{https://platform.openai.com/tokenizer}} recognizes "ephemeral" as a single subword rather than three morphemes ("ept", "hemera", "-al"), because the affixes are not explicitly segmented, and 'epheremal' is a rare word so BPE better learns it as a single token.

\appsubsection{Family of Fine-Tuning Methods} \label{app:finetuning}
Fine-tuning is the process of adapting a pre-trained model to a specific task T or domain D by updating its parameters on a target dataset \(\mathcal{D}\). This process is crucial for leveraging pre-trained models' knowledge and enhancing their performance on downstream tasks.

There are several approaches to fine-tuning, which can be categorized by: 1. the availability of labeled data (supervised vs unsupervised fine-tuning), 2. the extent of parameter updates (full-parameter vs parameter-efficient fine-tuning), and 3. task. We focus on supervised fine-tuning, which involves minimizing a task-specific loss function over a labeled dataset.

\appsubsubsection{Supervised Fine-Tuning (SFT)}\label{app:sft}

SFT involves adapting a pre-trained model to a target task by minimizing a task-specific loss function over a labeled dataset. For a dataset \( \mathcal{D} = \{(\mathbf{x}^{(i)}, \mathbf{y}^{(i)})\}_{i=1}^N \), where \( \mathbf{x}^{(i)} \) is the input and \( \mathbf{y}^{(i)} \) is the target output, the objective is to minimize:

\begin{equation}
\mathcal{L} = \frac{1}{N} \sum_{i=1}^N \ell(f(\mathbf{x}^{(i)}; \theta), \mathbf{y}^{(i)})
\end{equation}

where \( f(\mathbf{x}; \theta) \) represents the model's output with parameters \( \theta \), and \( \ell \) is the loss function, typically cross-entropy loss.

\appsubsubsection{Instruction tuning} \label{app:instruction-tuning-it}

Instruction-tuning is a specialized form of SFT \Cref{app:sft} where models are trained on datasets comprising instruction-response pairs. This approach enables models to generalize across various tasks described by natural language instructions, enhancing their ability to follow diverse prompts. Formally, an instruction-tuning dataset consists of pairs \( \{(\mathbf{I}^{(i)}, \mathbf{y}^{(i)})\}_{i=1}^N \) or triplets \( \{(\mathbf{I}^{(i)}, \mathbf{x}^{(i)}, \mathbf{y}^{(i)})\}_{i=1}^N \), where \( \mathbf{I}^{(i)} \) denotes the instruction, \( \mathbf{x}^{(i)} \) is the optional input, and \( \mathbf{y}^{(i)} \) is the desired output. The training objective is to minimize the loss:

\begin{equation}
\mathcal{L} = \frac{1}{N} \sum_{i=1}^N \ell(f(\mathbf{I}^{(i)}, \mathbf{x}^{(i)}; \theta), \mathbf{y}^{(i)})
\end{equation}

where \( f \) represents the model parameterized by \( \theta \), and \( \ell \) is the loss function measuring the discrepancy between the model's prediction and the target output.

\appsubsubsection{Parameter-Efficient Fine-Tuning} \label{app:peft}
Full-parameter fine-tuning updates \textit{all} parameters of a pre-trained model on the target dataset, which can be computationally expensive and memory-intensive for large models. Parameter-efficient fine-tuning (PEFT) methods adjust only a subset of the parameters, reducing computational and storage requirements while maintaining performance \citep{XuPARAMETEREFFICIENT2023}.

The most common PEFT method is Low-Rank Adaptation (LoRA), and its variants. They are used in the training process as a wrapper around the model's weights, allowing for efficient updates without modifying the entire model. This approach is particularly useful for large models, where full fine-tuning may be impractical due to resource constraints.

\paragraph{Low-Rank Adaptation (LoRA)} decomposes the weight updates into low-rank matrices, reducing the number of trainable parameters \citep{huLoRALowRankAdaptation2021}. Specifically, for a weight matrix \( W \in \mathbb{R}^{d \times k} \), LoRA introduces two low-rank matrices \( A \in \mathbb{R}^{d \times r} \) and \( B \in \mathbb{R}^{r \times k} \), where \( 0 < r \ll \min(d, k) \). The adapted weight is:

\begin{equation}
W' = W + \alpha \cdot A B
\end{equation}

Here, \( \alpha \) is a scaling factor that controls the contribution of the low-rank adaptation. The rank \( r \) determines the capacity of the adaptation, balancing between expressiveness and efficiency.

LoRA introduces \( 2dr \) trainable parameters (size of \( A \) and \( B \)), which is significantly smaller than the original \( dk \) parameters. This reduction in parameters enables efficient fine-tuning of large models on limited hardware. In practice, LoRA is applied to specific modules of the model, such as attention and feed-forward layers, to balance performance and efficiency.

\paragraph{Rank-Stabilized LoRA (rsLoRA)} modifies the scaling factor in LoRA to improve performance across different ranks. The standard scaling factor \( \gamma_r = \alpha / r \) can slow learning for higher ranks. rsLoRA proposes adjusting the scaling factor to \( \gamma_r = \alpha / \sqrt{r} \), enhancing fine-tuning performance without increasing inference costs.

\subsection{Reinforcement Learning (RL)} \label{app:rl}

Reinforcement Learning (RL) is a framework in which an agent interacts with an environment to learn a policy $\pi_\theta$ that maximizes a long-term reward. At each time step $t$, the agent observes a state, takes an action, and receives a reward $r_t$. The goal is to maximize the expected cumulative reward, given by

\begin{equation}
J(\theta) = \mathbb{E}_{\pi_\theta}\left( \sum_{t=0}^{T} \gamma^t\,r_t\right)
\end{equation}

where $\gamma\in(0,1)$ is a discount factor.

In the next section, we consider an advanced method called Group Relative Policy Optimization (GRPO). GRPO extends PPO by generating multiple responses per prompt, comparing the rewards within each group, and adjusting the policy based on relative advantages. This online RL approach continuously improves by (1) generating completions, (2) scoring them using reward models, and (3) updating the model's policy with both an advantage term and a KL penalty.

\appsubsubsection{Group Relative Policy Optimization} \label{app:grpo}

Group Relative Policy Optimization (GRPO) \citet{DeepSeek-AIDEEPSEEKR12025} is an online reinforcement learning method specifically designed for scenarios where the model generates multiple responses (or completions) for the same prompt. It was introduced to improve the mathematical reasoning capabilities of LLMs, by generating multiple CoT responses for a given problem and then compares results to the ground truth.

Intuitively, GRPO generates multiple responses for a given prompt, scores them using reward models, calculates the relative reward of the group, and then compares each response's score to that relative reward to determine which is better or worse. The model then updates its policy to favor high-reward responses.

\paragraph{Generating completions} For each prompt $q$ in a batch, the model generates a set of $G$ completions:
\begin{equation}
O_q = \{o_1, o_2, \ldots, o_G\}
\end{equation}

Each completion $o_i$ consists of a sequence of tokens:
\begin{equation}
o_i = \{o_{i,1}, o_{i,2}, \ldots, o_{i,|o_i|}\}
\end{equation}

\paragraph{Computing the advantage} For each completion, a reward $r_i$ is computed using predefined reward functions. To enable comparison within groups, the rewards are normalized:
\begin{equation}
\mu_r = \text{mean}(r)
\end{equation}
\begin{equation}
\sigma_r = \text{std}(r)
\end{equation}
\begin{equation}
\hat{A}_{i,t} = \frac{r_i - \mu_r}{\sigma_r}
\end{equation}

where $r = \{r_1, r_2, \ldots, r_G\}$ is the set of rewards for all completions in the group, and $\hat{A}_{i,t}$ is the advantage for token $t$ in completion $i$. This normalization gives the method its name: Group Relative Policy Optimization.

\paragraph{Estimating the KL divergence} To prevent the policy from deviating too far from the reference policy $\pi_{\text{ref}}$, the KL divergence is estimated:
\begin{equation}
\pi_\text{ratio} = \frac{\pi_\theta(o_{i,t} | q, o_{i,<t})}{\pi_{\text{ref}}(o_{i,t} | q, o_{i,<t})}
\end{equation}
\begin{equation}
\pi_\text{inv\_ratio} = \frac{\pi_{\text{ref}}(o_{i,t} | q, o_{i,<t})}{\pi_\theta(o_{i,t} | q, o_{i,<t})}
\end{equation}
\begin{equation}
D_{\text{KL}} = \log\pi_\text{ratio} - 1 + \pi_\text{inv\_ratio}
\end{equation}

\paragraph{Computing the loss} The GRPO objective combines the advantage term with a KL penalty:
\begin{equation}
L_{\text{adv}} = -\frac{1}{G}\sum_{i=1}^{G}\sum_{t=1}^{|o_i|}\pi_\text{ratio}\hat{A}_{i,t}
\end{equation}
\begin{equation}
L_{\text{KL}} = \beta D_{\text{KL}}
\end{equation}
\begin{equation}
L_{\text{GRPO}}(\theta) = L_{\text{adv}} - L_{\text{KL}}
\end{equation}

where $\beta$ is a hyperparameter that controls the weight of the KL penalty. The advantage term encourages the policy to assign higher probability to tokens that lead to better rewards, while the KL term ensures that the policy doesn't deviate too far from the reference policy.

\paragraph{Multiple updates} For multiple $\mu$ updates after each generation, GRPO uses a clipped surrogate objective. First, compute the old policy ratio:
\begin{align}
\pi_{\text{old\_ratio}} &= \frac{\pi_\theta(o_{i,t} \mid q, o_{i,<t})}{\pi_{\theta_{\text{old}}}(o_{i,t} \mid q, o_{i,<t})},
\end{align}
then clip it:
\begin{align}
\pi_{\text{clipped}} &= \text{clip}\Bigl(\pi_{\text{old\_ratio}},\, 1-\epsilon,\, 1+\epsilon\Bigr).
\end{align}
The clipped advantage loss is $L_{\text{adv\_clipped}}$
\begin{equation}
-\frac{1}{G}\sum_{i=1}^{G}\sum_{t=1}^{|o_i|}
\min\Bigl(\pi_{\text{old\_ratio}}\hat{A}_{i,t},\, \pi_{\text{clipped}}\hat{A}_{i,t}\Bigr),
\end{equation}
yielding the final objective:
\begin{equation}
L_{\text{GRPO\_clipped}}(\theta) = L_{\text{adv\_clipped}} - L_{\text{KL}}.
\end{equation}

Here, $\epsilon$ (small constant, typically 0.2) controls how much the policy can change in a single update and $\beta$ controls the KL penalty's strength.

In HuggingFace's \texttt{trl} library, GRPO is implemented in the \texttt{GRPOTrainer} class and number of updates $\mu$ is controlled by the \texttt{num\_iterations} parameter. The default value of $\mu = 1$ simplifies the objective to the original GRPO formulation.

  \input{appendix/config}
  \appsection{Costs} \label{app:cost}


  
\section{Documentation of previous iterations} \label{app:previous-iterations}
\subsection{Fine-tune OpenAI} \label{app:openai-finetune}

\subsection{Fine-tune Gemma-3-4b-it} \label{app:gemma-finetune}

For supervised fine-tuning (SFT), we utilized the \texttt{trl} library with the following hyperparameters: batch size \( b = 16 \), number of epochs \( \text{eps} = 4 \), learning rate \( \alpha = 2 \times 10^{-5} \), weight decay \( \lambda = 0.05 \), and a cosine annealing learning rate scheduler with restarts.

The batch size \( b \) defines the number of training examples processed simultaneously during each forward and backward pass. A batch size of 16 balances computational efficiency and gradient estimation accuracy. Training for 4 epochs (\( \text{eps} = 4 \)) means the model will see the training data a total of four times, which ensures sufficient exposure to the training data without risking overfitting. The learning rate \( \alpha \) controls the step size for weight updates; a value of \( 2 \times 10^{-5} \) is typical for fine-tuning large language models, facilitating gradual convergence. Weight decay \( \lambda \) serves as a regularization term, penalizing large weights to prevent overfitting. The cosine annealing scheduler adjusts the learning rate following a cosine decay pattern, periodically restarting to allow the model to escape local minima and potentially achieve better generalization, compared to linear decay.

\section{Reflection} \label{sec:reflection}

  \section{Minerva Appendix: LOs \& HCs} \label{sec:minerva}

\subsection{LOs} \label{sec:los}
\numpara{CS110-codeReadability} The codebase exemplifies best practices in Python programming, adhering strictly to PEP conventions. Each module is documented with detailed Google-style docstrings and descriptive inline comments to ensure that the logic behind functions and classes is transparent to collaborators and future users. By utilizing tools like Ruff for linting and formatting, and mypy for type-checking, the codebase achieves a consistent style and minimizes errors. Additionally, the inclusion of pre-commit hooks ensures that these standards are maintained across all contributions, fostering a robust and maintainable codebase.
\numpara{CS162-communication} The documentation strives to adhere to industry standards, by including an informative README, clear commit messages (using conventional commit or gitmoji), and documented pull request. Each module has a module-level docstring and function-level docstrings, to ensure that the code is understandable to users and collaborators. I used \verb|Ruff| (for linting and black-style code formatting) and \verb|mypy| (for type checking), to enforce consistent and error-free code presentation, to facilitate readability and maintainability for the author and future collaborators (if any).
\numpara{CS156-MLCode} The machine learning pipeline was designed in Python to be both functional and comprehensible. The code integrates data processing, model fine-tuning, and hyperparameter tuning into a seamless pipeline, with each step working and documented. Model evaluation are explicitly defined in appendix \Cref{app:training-details}, ensuring replicability.
\numpara{CS156-MLExplanation} Currently, most details and explanations are defined in section \Cref{app:technicality} and appendix \Cref{app:training-details}. The final paper will provide more high-level diagrams of the machine learning techniques used in fine-tuning the Gemma-2 model. The supporting diagrams will visualize key processes, such as hyperparameter tuning and model evaluation. It will ensure that the methodology and results are accessible to a less specialized audience.
\numpara{CS162-separationofconcerns} The codebase is organized into distinct Python modules, each focused on a specific task such as data processing, mnemonic processing, and model fine-tuning. This separation of concerns aligns with best practices in software design, ensuring that each function is highly cohesive and performs a single well-defined responsibility. By maintaining modularity, the codebase facilitates easier debugging, testing, and future scaling, contributing to its long-term maintainability and effectiveness.

% TODO: Capstone LOs
\subsection{Capstone LOs} \label{sec:capstone-los}
\numpara{navigation} \Cref{app:previous-iterations}
\numpara{outcomeanalysis}
\numpara{curation}
\numpara{qualitydeliverables}

\subsection{HCs} \label{sec:hcs}
\numpara{audience} The project addresses dual audiences: (1) NLP researchers investigating LLMs' linguistic capabilities, for whom we provide detailed technical methodologies and evaluation metrics; and (2) educational technology developers seeking practical approaches to vocabulary learning assistance, for whom we demonstrate application potential and implementation strategies. For example, \Cref{sec:mnemonic-review} introduces key concepts about mnemonic devices and language teaching with sufficient depth for both audiences, while hiding technical details in appendices but providing pointers to those details in each section for interested readers. The paper's structure and content are designed to facilitate knowledge transfer across these domains, ensuring that both audiences can derive value from the research findings. Technical content is balanced with practical implications for vocabulary acquisition and language learning, ensuring relevance to both research and application-oriented readers.

\numpara{organization} The paper follows standard ACL formatting conventions, with a logical progression from theoretical foundations through methodology to empirical results. The structure facilitates efficient information extraction, with each section building upon previous content. Key contributions are identified early (\Cref{sec:intro}) and systematically developed throughout subsequent sections. Technical details that might interrupt argumentative flow are relegated to appendices, maintaining narrative coherence while ensuring methodological transparency for reproduction purposes. This organization aligns with expectations of the computational linguistics community while supporting efficient knowledge transfer.

\numpara{gapanalysis} \Cref{sec:intro} identifies critical limitations in existing mnemonic generation approaches: (1) overreliance on the keyword method, which fails for abstract vocabulary lacking concrete referents, and (2) neglect of the rich linguistic knowledge embedded in LLMs that could enable more diverse mnemonic strategies beyond simple keyword associations. Prior work has also passively delivered mnemonics to learners rather than leveraging individual learning preferences, despite research showing self-created mnemonics enhance retention. These identified gaps motivate our project to elicit linguistic reasoning and creativity from LLMs, enabling them to generate mnemonics that are not only effective but also linguistically grounded. Our final model, \linksys, after deployment, could interact with learners to tailor mnemonic generation based on their preferences, enhancing the learning experience. This approach addresses the limitations of existing methods by providing a more comprehensive and personalized vocabulary learning tool. The project also contributes to the field of educational technology by exploring how LLMs can be harnessed for creating effective resources for language education and self-study, such as creating mnemonic devices.

% TODO
\numpara{hypothesisdevelopment} The research questions in \Cref{sec:intro} establish testable hypotheses regarding LLMs as linguistic knowledge bases for mnemonic generation. We hypothesize that fine-tuning LLMs on linguistically annotated examples will improve mnemonic quality across semantic relevance, diversity, and helpfulness dimensions. This hypothesis is predicated on the theoretical assumption that LLMs encode significant linguistic knowledge during pre-training that can be accessed through targeted fine-tuning. The hypothesis is operationalized through specific evaluation metrics described in \Cref{sec:evaluation}, ensuring empirical testability.

\numpara{scienceoflearning}

\numpara{optimization} ...

\numpara{algorithms} The fine-tuning methodology detailed in \Cref{app:finetuning} incorporates algorithmic innovations in parameter-efficient adaptation. Specifically, we implement QLoRA with rank-stabilized scaling (rsLoRA), modifying the standard scaling factor to improve performance across different ranks. This algorithm reduces memory requirements by quantizing the pre-trained model to 4-bit precision while allowing selective updates to low-rank adaptation matrices. Population-based training explores the hyperparameter space dynamically, pruning underperforming configurations and exploring promising regions to optimize validation performance.

\numpara{heuristics} Our approach employs several problem-solving heuristics to enhance mnemonic generation. The linguistic feature classification system provides a structured framework for identifying and leveraging different linguistic aspects in vocabulary. We use problem decomposition by separating mnemonic creation into linguistic analysis and creative association phases, enabling more systematic knowledge utilization. Visualization techniques in the form of embedding space projections help identify semantic relationships between vocabulary terms and potential mnemonic content. These heuristics guide both the model fine-tuning process and the subsequent evaluation methodology.

\numpara{sampling} The evaluation methodology employs stratified random sampling to ensure representation across linguistic feature categories (\Cref{sec:vocab-selection}) and vocabulary complexity levels. For human evaluation, we randomly selected 50 test examples, stratified by linguistic feature, to obtain less biased assessments of mnemonic quality. This sampling strategy ensures balanced representation of different mnemonic types while maintaining statistical validity. Each example received multiple independent ratings to mitigate individual rater bias, with inter-rater reliability calculated using Cohen's Kappa score to confirm consistency in ratings.

\numpara{dataviz} We employ targeted data visualizations to communicate complex relationships between linguistic features and mnemonic effectiveness. Radar charts display the distribution of linguistic features across different model outputs, while heatmaps visualize correlation patterns between computational metrics and human evaluations. Embedding space projections illustrate semantic relationships between vocabulary terms and their mnemonics, providing intuitive visual confirmation of semantic relevance scores. These visualizations enhance interpretability of results while supporting our conclusions regarding the contribution of different linguistic features to mnemonic quality.

\numpara{significance} Statistical significance testing (\Cref{sec:qualitative-llm-judge}) confirms the reliability of our comparative results between baseline and \linksys. We employ paired statistical tests (Wilcoxon signed-rank and McNemar test) to account for vocabulary-specific variation when comparing model outputs on identical test sets. Effect size calculations quantify the practical significance of improvements, while confidence intervals provide transparency regarding the precision of our estimates. Multiple comparison corrections maintain statistical rigor when evaluating performance across different linguistic feature categories.

\numpara{shapingbehavior} The intended application of our approach shapes vocabulary learning behavior by encouraging deeper engagement with linguistic features. Rather than keyword method, the generated mnemonics prompt learners to recognize morphological, etymological, and semantic patterns, fostering more robust mental representations. By explicitly highlighting these linguistic features, the system promotes analytical processing of vocabulary, which research indicates enhances long-term retention. This behavior-shaping aspect represents a significant advantage over keyword-only approaches that rely on shallow phonetic or orthographic associations.

\numpara{biasmitigation}

\numpara{ethicalconsiderations} Our work addresses ethical considerations in educational technology deployment. We explore linguistic diversity by analyzing and leveraging several linguistic features for mnemonic generation, ensuring that the generated mnemonics are inclusive and accessible to learners from diverse linguistic backgrounds. The model is trained on a diverse dataset of vocabulary words, including those from various languages and etymological origins, to ensure that the mnemonics generated are relevant and culturally sensitive. We also consider the potential for bias in the generated mnemonics by ensuring that the training data is representative of a wide range of vocabulary words.

\numpara{studyreplication} To facilitate replication, we provide comprehensive implementation details in \Cref{app:training-details}, including environment setup, model parameters, and training configurations. The dataset construction process is documented in \Cref{sec:data-gen}, with preprocessing steps explicitly specified. All code and datasets are made publicly available through GitHub and HuggingFace repositories, with standardized formats ensuring compatibility with common ML frameworks. This transparency ensures that other researchers can validate the findings and build upon the methodology.

\end{appendices}

\end{document}
